% \textcolor{red}{\lipsum[10-13]}
This thesis studied the key properties of native point defects in wurtzite ZnO, such as  defect energetics, electronic band structure and density of states evaluated via first principle calculations using Density Functional Theory (DFT). All calculations were done using the open source Quantum ESPRESSO software. Hubbard-U correction scheme was applied to correct the inherent band gap underestimation of standard DFT calculations. Convergence of the total energy and thermodynamic pressure were tested against the k-points used, and the energy cutoffs used in the wavefunction and in the electron density.
A $3 \times 3 \times 2$ supercell was constructed from the unit cell of the wurtzite ZnO to mimic the periodic boundary conditions of a bulk ZnO. Results showed that defects associated with oxygen excess  or zinc deficiency are generally electron acceptors while zinc excess or oxygen deficiency are generally electron donors.  Zinc vacancy, oxygen interstitial, and oxygen antisite were shown to be acceptor-type defects based on band structure calculations. On the other hand, oxygen vacancy, zinc interstitial, and zinc antisite were shown to be donor-type defects. The non-stoichiometry of ZnO are attributed to the low defect formation energies of oxygen vacancy and zinc interstitial in the zinc-rich environment, while zinc vacancy in the oxygen-rich environment. The observed unintentional $n$-type conductivity in ZnO cannot be attributed to donor-type defects since oxygen vacancy is a deep donor while zinc vacancy and zinc interstitial are unstable due to high formation energies. However, they can serve as potential source of hole compensation in $p$-type ZnO. Further studies on defect simulations by incorporating doping of impurity atoms and surface defect states are needed to fully describe the defect interactions present in wurtzite ZnO.  

\begingroup
\noindent
\textbf{PACS:} 71.15.Mb Density functional theory, local density approximation, gradient and other corrections; 71.20.-b Electron density of states and band structure of crystalline solids; 71.55.-i Impurity and defect levels
\endgroup