\chapter{THEORETICAL FRAMEWORK}
\section{Electronic Structure}
The problem of electronic structure methods begins with the attempt to solve the general non-relativistic time-independent Schr\"{o}dinger equation given as 
    \begin{equation} \label{eq:schrodinger}
        \hat{\mathcal{H}} \Psi = E \Psi
    \end{equation}
where $\hat{\mathcal{H}}$ is the Hamiltonian operator for a system of electrons, $\Psi$ is the electronic wavefunction and $E$ is the energy of the system. Consider a single electron in three dimensional system, the Schr\"{o}dinger equation can be expressed as 
    \begin{equation} \label{eq:1e_wave}
        \hat{\mathcal{H}} \Psi_n = - \frac{\hbar^2}{2m} \left(\pdv[2]{x} + \pdv[2]{y} + \pdv[2]{z} \right) \Psi_n + V \Psi_n  = \epsilon_n \Psi_n
    \end{equation}
where $m$ is the mass of electron, $V$ is the effective potential energy and $\epsilon_n$ is the energy of electron in the orbital. The term orbital denotes the solution of the Schr\"{o}dinger equation for a system of only one electron. This will be useful in later sections because this will allow to distinguish between the exact quantum state of a system of $N$ interacting  electrons 
from the approximate quantum state of $N$ electrons in $N$ orbitals, where each orbital is a solution to one-electron wavefunction in \eqref{eq:1e_wave}. If $V$ is zero for the case of free electrons (i.e. non-interacting), then the orbital model is exact. 
 
Since electrons are restricted by the potential inside the atom, the simplest way of solving \eqref{eq:1e_wave} is by considering an infinite potential well. The electrons are confined inside a cube of length $L$ where the potential $V$ inside is zero and infinite at outside must satisfy the boundary condition

\begin{equation}
	\Psi_n(L_x,L_y,L_z) = 0	
\end{equation}
where $L_x,L_y,L_z$ can be either 0 or $L$. The solution will have a sine dependence

\begin{equation}
	\Psi_n(x,y,z) = \sqrt{\left(\frac{2}{L}\right)^3} \ \sin(\frac{n_x \pi }{L} x) \sin(\frac{n_y \pi}{L} y) \sin(\frac{n_z \pi}{L} z)
\end{equation}
where $n_x,n_y,n_z$ are integer quantum states. Provided that $ k_i = n_i \pi / L$ where $i=x,y, \text{or}\, z$; then the energy dispersion relation can be expressed as 
\begin{equation} \label{eq:free_e}
    \epsilon_k = \frac{\hbar^2}{2m} (k_x^2 + k_y^2 + k_z^2) = \frac{\hbar^2}{2m} k^2 \propto k^2
\end{equation}
Note that energy levels are discretized by the quantum states which arises from imposing the boundary conditions. 
    \subsection{Electronic Band structure}
    Inside the crystal lattice, the periodic arrangement of atoms or ions causes the potential to be periodic which eventually gives rise to the formation of energy bands. The wavefunction $\Psi$ will become periodic in space with a period $L$ and must obey the Born-von Karman boundary condition 
\begin{equation} \label{eq:periodic}
    \Psi_k(x,y,z) = \Psi_k(x + L, y, z) 
\end{equation}
and similarly for the $y$ and $z$ coordinates. It can be shown that wavefunctions satisfying \eqref{eq:1e_wave} and \eqref{eq:periodic} are the Bloch form of a travelling plane wave
\begin{equation} \label{eq:Bloch}
    \Psi_k(\va{r}) = u_k(\va{r}) \exp(i \va{k} \vdot \va{r})
\end{equation}  
where $u_k(\va{r})$ has the period of the crystal lattice with $u_k(\va{r}) = u_k(\va{r} + \va{R})$. Here $\va{R}$ is the translation vector which can be simply thought as the periodicity expressed as vector.  The Bloch expression can be written as
\begin{align}
    \Psi_k(\va{r} + \va{R}) &= u_k(\va{r} + \va{R}) \exp(i \va{k} \vdot (\va{r} + \va{R}))  \notag \\
    \Psi_k(\va{r} + \va{R}) &= u_k(\va{r}) \exp(i \va{k} \vdot \va{r})  \exp(i \va{k} \vdot \va{R}) \notag \\
    \Psi_k(\va{r} + \va{R}) &= \Psi_k(\va{r}) \exp(i \va{k} \vdot \va{R}) 
\end{align}
Notice that the wavefunction differs from the plane wave of free electrons only by a periodic modulation given by the new phase factor. This means that the electrons in the crystal lattice are treated as perturbed weakly by the periodic potential of the ion cores.

% in which the components of $\va{k}$ satisfy
% \begin{equation}
%     k_i = \pm \frac{2n_i \pi}{L} \quad ; i = x,y,z \quad; n_i = 0, 1, 2, \dots
% \end{equation}

        \subsubsection{Band structure of free electron}
            A special case of periodicity is where the potential is set to zero, which is applicable for the free electrons. The wavefunction will be a plane wave 
            \begin{equation}
                \Psi_k(\va{r})  = \exp(i \va{k} \vdot \va{r})
            \end{equation} 
        that represents travelling wave with a momentum $\va{p} = \hbar \va{k}$. The energy dispersion relation is still given by \eqref{eq:free_e} but this time the allowed energy values are distributed essentially from zero to infinity. Figure \ref{fig:free-electron} shows the parabolic dependence of energy with the wavevector $k$. Since the system is periodic in real space, it must be true for the reciprocal space, in this case by $2\pi/a$ where $a$ is some lattice constant. Figure \ref{fig:free-electron}a shows the extended zone scheme where there are no restrictions on the values of wavevector $\va{k}$. When wavevectors are outside the first Brillouin zone (BZ), they can be translated back to the first zone by subtracting a suitable reciprocal lattice vector. In mathematical sense
        \begin{equation} \label{eq:band_fold}
            \va{k} + \va{G} = \va{k'}
        \end{equation}
        where $\va{k'}$ is the unrestricted wavevector, $\va{k}$ is in the first Brillouin zone, and $\va{G}$ is the translational reciprocal lattice vector. The energy dispersion relation can always be written as 
        \begin{align}
            \epsilon(k_x,k_y,k_z) &= \frac{\hbar^2}{2m} (\va{k}+\va{G})^2 \notag \\
                                &= \frac{\hbar^2}{2m}[(k_x + G_x)^2 + (k_y + G_y)^2 + (k_z + G_z)^2]
        \end{align}
        Figure \ref{fig:free-electron}b shows the reduced zone scheme where the  bands are folded into the first BZ by applying \eqref{eq:band_fold}. Any energy state beyond the first BZ is the same to a state inside the first BZ with a different band index $n$.

 \begin{figure}[tbh!]
	\centering
	\includegraphics[width=0.7\linewidth]{"images/free electron"}
	\caption[Free electron band structure]{Free electron band structure}
	\label{fig:free-electron}
\end{figure}

\subsubsection{Band structure of electrons in solids}
When atoms are very far from each other with no interaction, each electron occupies specific discrete orbitals such as 1s, 2p, 3d, etc. When they are bring  closer enough, the outermost (valence) electrons interact with each other and will result in the  energy level splitting. The innermost (core) electrons remain as they are, since they are closer to the nuclei and bounded by a deep potential well. For a solid containing a large $N$ atoms, there will be $N$ orbitals (i.e. $N$ 3d-orbitals) trying to occupy the same energy level. Pauli's exclusion principle will prevent this from happening, hence what happens is there will be splitting of the energy level that are closely spaced and this will eventually form a continuous band of energy levels. Figure \ref{fig:band_model} summarizes the evolution of energy levels as the atoms are brought together.

\begin{figure}[tbh!]
	\centering
	\includegraphics[width=0.7\linewidth]{"images/band model"}
	\caption[Band structure in solids]{Formation of bands and band gaps when isolated atoms are bring closer together}
	\label{fig:band_model}
\end{figure}    

Another interesting property of band structure is the formation of energy band gaps. This happens when the valence electrons interact with the periodic potential of the nuclei. Assuming a weak periodic potential, most of the band structure will not changed very much, except possibly at the Brillouin zone boundaries with a wavevector of $\va{k} = n \pi/ a$. The orbitals with the wavevector at zone boundaries, chosen to be at high symmetry points, follows the Bragg diffraction condition and thus are diffracted. The valence electrons are scattered (or reflected) at the zone boundary in which the wavefunction are made up of equal plane waves travelling from the left and from the right. The wavefunction becomes a standing wave that resembles more of those bound states. Hence, there will be a forbidden region where travelling waves are not allowed. If sufficient energy is provided to the electron, they can overcome the  binding potential.

The band gap is generally referred to the energy difference between the top of valence band, Valence band maximum (VBM),  and the bottom of the conduction band, Conduction band minimum (CBM). If VBM and CBM coincides with each other, the material is said to be a conductor. Electrons can easily occupy the conduction band without any excitation, hence electrons are highly mobile that will lead to high current. For band gaps with a value comparable to the quantity $k_B T$, where $k_B$ is the Boltzmann constant and  $T$ is the absolute temperature near room temperature, then the material is semiconductor. If band gap is much larger than $k_B T$, then the material is insulator. However, this criterion is very loose because there are materials with large band gaps such as $ZnO$, $SrIn2O4$, that are categorized as semiconductors. These materials are generally called wide-band gap semiconductors. If the VBM and CBM are located in the same wavevector $k$, then the gap is direct. Otherwise, it is indirect. 

    {\color{red} insert the symmetry points in IBZ.}

    \subsection{Density of States}
    Another useful quantity in describing the electronic structure is the density of states (DOS). In general, the density of states can be defined as 
    \begin{equation} \label{eq:dos_sum}
        D(\epsilon) = 2 \sum_n \sum_k \delta(\epsilon - \epsilon_n(k))
    \end{equation}
    where for each band index $n$, the sum is over all allowed values of $k$ lying inside the first Brillouin zone. The factor 2 comes from the allowed values of the spin quantum number for each allowed value of $k$. In the limit of large crystal, the $k$ points are very close together, and the sum can be replaced by an integral. Since each allowed states will take up a volume of $ (\Delta k)^3 = \pi^3/V$ where $V$ is the volume of the solid in real space, it is convenient to write \eqref{eq:dos_sum} as 
    \begin{equation} \label{eq:dos_int}
        D(\epsilon) = 2\, \frac{V}{\pi^3} \sum_n \sum_k \delta(\epsilon - \epsilon_n(k)) (\Delta k)^3
    \end{equation}
    for  in the limit of $V \rightarrow \infty $, $\Delta k \rightarrow 0$, it becomes

    \begin{equation}
        \lim_{V \to \infty} \frac{1}{V}\, D(\epsilon) = \frac{2}{\pi^3} \sum_n \int \delta(\epsilon - \epsilon_n(k)) \dd[3]{k}
    \end{equation}  
    Usually, the total DOS is set to be the number of states per unit energy per unit volume. 

    The DOS can be projected in terms of the orbital contribution of each atoms. This can be expanded in a complete orthonormal basis as

    \begin{align}
        D(\epsilon) &= \sum_i D_i(\epsilon) \\
        &= \sum_i \sum_n \int \bra{\psi_n} i \ket{\psi_n}  \delta(\epsilon - \epsilon_n(k)) \dd[3]{k}
    \end{align}
    where $D_i(\epsilon)$ is the projected density of states (PDOS) of orbital $i$. 
     
\section{Many-body Quantum Mechanics}
insert text here
    \subsection{Time Independent Schr{\"o}dinger Equation}
    \subsection{Simplifying Assumptions}
    \subsection{Use of Atomic Units}
    \subsection{Hamiltonian Operator}
\subsection{Indistinguishability of electrons}
\section{Early First Principle Calculations}
    \subsection{n-electron problem}
    \subsection{Hartree Method}
    \subsection{Hartree-Fock Method}
\section{Density Functional Theory}
    \subsection{Electron Density}
    \subsection{Hohenberg-Kohn (HK) Formalism}
    \subsubsection{First HK Theorem}
        \subsubsection{Second HK Theorem}
    \subsection{Kohn Sham (KS) Formalism}
        \subsubsection{KS Equation}
        \subsubsection{Energy Terms}
    \subsection{Self Consistent Field Calculation}
\section{Exchange-correlation Functional}
    \subsection{Local Density Approximation (LDA)}
    \subsection{Generalized Gradient Approximation (GGA)}

\section{Corrections to DFT}
    \subsection{GW Method}
    \subsection{Hybrid Functionals}
    \subsection{Hubbard U Correction}

Example of double quotes ``word''. Lorem ipsum dol