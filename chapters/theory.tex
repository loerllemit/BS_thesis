\chapter{THEORETICAL FRAMEWORK}
\section{Electronic Structure}
The problem of electronic structure methods begins with the attempt to solve the general non-relativistic time-independent Schr\"{o}dinger equation given as 
    \begin{equation} \label{eq:schrodinger}
        \hat{\mathcal{H}} \Psi = E \Psi
    \end{equation}
where $\hat{\mathcal{H}}$ is the Hamiltonian operator for a system of electrons, $\Psi$ is the electronic wavefunction and $E$ is the energy of the system. Consider a single electron in three dimensional system, the Schr\"{o}dinger equation can be expressed as 
    \begin{equation} \label{eq:1e_wave}
        \hat{\mathcal{H}} \Psi_n = - \frac{\hbar^2}{2m} \left(\pdv[2]{x} + \pdv[2]{y} + \pdv[2]{z} \right) \Psi_n + V \Psi_n  = \epsilon_n \Psi_n
    \end{equation}
where $m$ is the mass of electron, $V$ is the effective potential energy and $\epsilon_n$ is the energy of electron in the orbital. The term orbital denotes the solution of the Schr\"{o}dinger equation for a system of only one electron. This will be useful in later sections because this will allow to distinguish between the exact quantum state of a system of $N$ interacting  electrons 
from the approximate quantum state of $N$ electrons in $N$ orbitals, where each orbital is a solution to one-electron wavefunction in \eqref{eq:1e_wave}. If $V$ is zero for the case of free electrons (i.e. non-interacting), then the orbital model is exact. 
 
Since electrons are restricted by the potential inside the atom, the simplest way of solving \eqref{eq:1e_wave} is by considering an infinite potential well. The electrons are confined inside a cube of length $L$ where the potential $V$ inside is zero and infinite at outside must satisfy the boundary condition

\begin{equation}
	\Psi_n(L_x,L_y,L_z) = 0	
\end{equation}
where $L_x,L_y,L_z$ can be either 0 or $L$. The solution will have a sine dependence

\begin{equation}
	\Psi_n(x,y,z) = \sqrt{\left(\frac{2}{L}\right)^3} \ \sin(\frac{n_x \pi }{L} x) \sin(\frac{n_y \pi}{L} y) \sin(\frac{n_z \pi}{L} z)
\end{equation}
where $n_x,n_y,n_z$ are integer quantum states. Provided that $ k_i = n_i \pi / L$ where $i=x,y, \text{or}\, z$; then the energy dispersion relation can be expressed as 
\begin{equation} \label{eq:free_e}
    \epsilon_k = \frac{\hbar^2}{2m} (k_x^2 + k_y^2 + k_z^2) = \frac{\hbar^2}{2m} k^2 \propto k^2
\end{equation}
Note that energy levels are discretized by the quantum states which arises from imposing the boundary conditions. 
    \subsection{Electronic Band structure}
    Inside the crystal lattice, the periodic arrangement of atoms or ions causes the potential to be periodic which eventually gives rise to the formation of energy bands. The wavefunction $\Psi$ will become periodic in space with a period $L$ and must obey the Born-von Karman boundary condition 
\begin{equation} \label{eq:periodic}
    \Psi_k(x,y,z) = \Psi_k(x + L, y, z) 
\end{equation}
and similarly for the $y$ and $z$ coordinates. It can be shown that wavefunctions satisfying \eqref{eq:1e_wave} and \eqref{eq:periodic} are the Bloch form of a travelling plane wave
\begin{equation} \label{eq:Bloch}
    \Psi_k(\va{r}) = u_k(\va{r}) \exp(i \va{k} \vdot \va{r})
\end{equation}  
where $u_k(\va{r})$ has the period of the crystal lattice with $u_k(\va{r}) = u_k(\va{r} + \va{R})$. Here $\va{R}$ is the translation vector which can be simply thought as the periodicity expressed as vector.  The Bloch expression can be written as
\begin{align}
    \Psi_k(\va{r} + \va{R}) &= u_k(\va{r} + \va{R}) \exp(i \va{k} \vdot (\va{r} + \va{R}))  \notag \\
    \Psi_k(\va{r} + \va{R}) &= u_k(\va{r}) \exp(i \va{k} \vdot \va{r})  \exp(i \va{k} \vdot \va{R}) \notag \\
    \Psi_k(\va{r} + \va{R}) &= \Psi_k(\va{r}) \exp(i \va{k} \vdot \va{R}) 
\end{align}
Notice that the wavefunction differs from the plane wave of free electrons only by a periodic modulation given by the new phase factor. This means that the electrons in the crystal lattice are treated as perturbed weakly by the periodic potential of the ion cores.

% in which the components of $\va{k}$ satisfy
% \begin{equation}
%     k_i = \pm \frac{2n_i \pi}{L} \quad ; i = x,y,z \quad; n_i = 0, 1, 2, \dots
% \end{equation}

        \subsubsection{Band structure of free electron}
            A special case of periodicity is where the potential is set to zero, which is applicable for the free electrons. The wavefunction will be a plane wave 
            \begin{equation}
                \Psi_k(\va{r})  = \exp(i \va{k} \vdot \va{r})
            \end{equation} 
        that represents travelling wave with a momentum $\va{p} = \hbar \va{k}$. The energy dispersion relation is still given by \eqref{eq:free_e} but this time the allowed energy values are distributed essentially from zero to infinity. Figure \ref{fig:free-electron} shows the parabolic dependence of energy with the wavevector $k$. Figure \ref{fig:free-electron}a shows the extended scheme

\begin{figure}[tbh!]
	\centering
	\includegraphics[width=0.7\linewidth]{"images/free electron"}
	\caption[]{Free electron band structure}
	\label{fig:free-electron}
\end{figure}




        insert the symmetry points in IBZ.
    \subsection{Density of States}
    explains fermi dirac distribution
    \subsection{Projected Density of States}
\section{Many-body Quantum Mechanics}
insert text here
    \subsection{Time Independent Schr{\"o}dinger Equation}
    \subsection{Simplifying Assumptions}
    \subsection{Use of Atomic Units}
    \subsection{Hamiltonian Operator}
\subsection{Indistinguishability of electrons}
\section{Early First Principle Calculations}
    \subsection{n-electron problem}
    \subsection{Hartree Method}
    \subsection{Hartree-Fock Method}
\section{Density Functional Theory}
    \subsection{Electron Density}
    \subsection{Hohenberg-Kohn (HK) Formalism}
    \subsubsection{First HK Theorem}
        \subsubsection{Second HK Theorem}
    \subsection{Kohn Sham (KS) Formalism}
        \subsubsection{KS Equation}
        \subsubsection{Energy Terms}
    \subsection{Self Consistent Field Calculation}
\section{Exchange-correlation Functional}
    \subsection{Local Density Approximation (LDA)}
    \subsection{Generalized Gradient Approximation (GGA)}

\section{Corrections to DFT}
    \subsection{GW Method}
    \subsection{Hybrid Functionals}
    \subsection{Hubbard U Correction}

Example of double quotes ``word''. Lorem ipsum dol