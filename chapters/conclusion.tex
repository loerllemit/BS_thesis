\chapter{Conclusion} \label{chap:conclu}
Defects associated with Zn excess or O deficiency are expected to be donor types since the former introduces two available electrons to donate while the latter forms a dangling bond where lone electrons are in the Zn atom. Zinc atoms prefer to loose electrons to attain their cationic state. On the other hand, defects associated with O excess or Zn deficiency are expected to be acceptor type defects. The excess \ch{O} atom will accept electrons to complete its electronic shell while deficient \ch{Zn} atom will cause a shortage of electrons hence  dangling bonds will form in the neighboring O atoms. O atoms prefer to accept electrons to attain their anionic state. Hence, oxygen vacancy, zinc interstitial, and zinc antisites were shown to be donor-type defects based on the bandstructure calculations done in this study. On the other hand, zinc vacancy, oxygen interstitial, and oxygen antisites were shown to accept-type defects. 





