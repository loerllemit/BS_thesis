\chapter{Conclusion and Outlook} \label{chap:conclu}
\vspace{-1em}
Defects associated with Zn excess or O deficiency are expected to be donor types since the former introduces two available electrons to donate while the latter forms a dangling bond where lone electrons are in the Zn atom. Zinc atoms prefer to loose electrons to attain their cationic state. Hence, this defect type will form  positive charge states. On the other hand, defects associated with O excess or Zn deficiency are expected to be acceptor type defects. The excess \ch{O} atom will accept electrons to complete its electronic shell while deficient \ch{Zn} atom will cause a shortage of electrons hence  dangling bonds will form in the neighboring O atoms. O atoms prefer to accept electrons to attain their anionic state.
This defect type will form  negative charge states. Hence, oxygen vacancy, zinc interstitial, and zinc antisite were shown to be donor-type defects based on the bandstructure calculations done in this study. On the other hand, zinc vacancy, oxygen interstitial, and oxygen antisite were shown to be acceptor-type defects.

Oxygen vacancy,  zinc interstitial and zinc antisite cannot be responsible for the observed unintentional $n$-type conductivity in wurtzite ZnO since oxygen vacancy is a deep donor, hence it cannot be a source of carrier electrons, while  zinc interstitial and zinc antisite have high defect formation energies. However, they can serve as a potential source of hole compensation in  $p$-type material since they have low formation energies. The non-stoichiometry in ZnO are attributed to the low defect formation energies of oxygen vacancies and  zinc interstitials in the zinc-rich limit and $p$-type material, while zinc vacancies in the oxygen-rich limit and $n$-type material.

In this thesis, only a small part of all possible defects was done.  A vast amount of defects and defect complexes should be investigated in order to see the whole picture of the defect interactions present in wurtzite ZnO. This includes doping of impurity atoms and defects present in surface boundaries. 


