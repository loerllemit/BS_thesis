\chapter{Results and Discussion} \label{chap:rnd}
\section{Convergence Tests}
Convergence tests on the ZnO unit cell were done by varying the k-points, and the kinetic energy cut-offs  of the Kohn-Sham orbital and the electron density, respectively.  As shown in Figure \ref{fig:conv_k}, the calculated pressure and total energy converges at $k \geq 2$ which corresponds to the Monkhorst-Pack grid of $7 \times 7 \times 2$. One can choose this minimum k-point required for convergence to be used for subsequent calculations. However, higher values are preferred as to make sure the convergence is guaranteed for all configuration of the system. In this thesis, the k-point used was set $k = 6$ which corresponds to the Monkhorst-Pack grid of $11 \times 11 \times 6$. Next, the cutoff energy of the Kohn-Sham orbitals $E_{\text{wfc}}$ were varied until convergence was achieved. At $E_{\text{wfc}} = 40$ Ry, very small fluctuations are observed for both the total energy and pressure as verified in Figure \ref{fig:conv_wfc}. However, $E_{\text{wfc}} = 90$ Ry was chosen so that it can take into account the possible errors due to supercell image interaction and spurious interaction of defects \citep{White1994,Makov1995,Martyna1999}. Lastly, the cutoff energy of the electron density $E_{\text{rho}}$ converged at 720 Ry corresponding to the dual value of 8 ($8 \times E_{\text{wfc}}$) as illustrated in Figure \ref{fig:conv_rho}. The values of $E_{\text{wfc}} = 90$ Ry, $E_{\text{rho}} = 720$ Ry, and Monkhorst-Pack of $11 \times 11 \times 6$ will be used as convergence parameters for all supercell simulations.

\begin{figure}[tbh!]
	\centering
	\includegraphics[width=0.7\linewidth]{"images/rnd/kpoints"}
	\caption[Convergence of pressure and total energies with respect to k-points]{Convergence of pressure and total energies with respect to k-points. The x-axis denotes the k-point $k$ in the Monkhorst-Pack grid of $(k+5) \times (k +5 ) \times k$}
	\label{fig:conv_k}
\end{figure}

\begin{figure}[tbh!]
	\centering
	\includegraphics[width=0.7\linewidth]{"images/rnd/ecutwfc"}
	\caption[Convergence of pressure and total energies with respect to cutoff energy of the Kohn-Sham orbital]{Convergence of pressure and total energies with respect to cutoff energy of the Kohn-Sham orbital $E_{\text{wfc}}$.}
	\label{fig:conv_wfc}
\end{figure}

\begin{figure}[tbh!]
	\centering
	\includegraphics[width=0.7\linewidth]{"images/rnd/ecutrho"}
	\caption[Convergence of pressure and total energies with respect to cutoff energy of the electron density]{Convergence of pressure and total energies with respect to cutoff energy of the electron density $E_{\text{rho}}$.}
	\label{fig:conv_rho}
\end{figure}

\section{Hubbard-U parameters}
The influence of both $U_p$, acting on O 2p orbitals,  and $U_d$, acting on Zn 3d orbitals, on the p-d hybridization is evident in the variation of band gap as shown in Figure \ref{fig:hubbard_val}. For a fixed $U_d$, there is a huge increase in the magnitude of the electronic band gap of almost 0.2 eV with the increment of $U_p$ by 0.5 eV. The rate of increase in band gap is high enough that most band gap values have exceeded the reference band gap, shown as a horizontal dashed line in Figure \ref{fig:hubbard_val}.  On the other hand,  the band gap increases steadily with $U_d$ for a given $U_p$. The higher the $U_d$ value, the lower $U_p$ needed to widen the band gap. In either case, the band gap monotonically increase with both parameters. The observed trend agrees very well with the recent results of Goh et al. \citep{Goh2017}.


Among the values, $U_d = 15$ eV and $U_p = 7$ eV has the band gap  value of 3.395 eV which is closest to the experimental value of 3.37 eV. Other combinations of $U_d$ and $U_p$ listed in Table \ref{tab:hubbard_table} are equally valid in correcting the band gap. Nevertheless, this study chose $U_d = 15$ eV and $U_p = 7$ eV as the correction parameters since the calculated band gap falls between the experimental band gap of 3.37 eV to 3.44 eV. Standard DFT calculations using GGA will produce severe underestimation of the band gap with  a value of 0.7589 eV, an error of more than 75\%, consistent with the reports of Haq et al. \citep{Haq2013} and Erhart et al. \citep{Erhart2006a}. This underestimation will affect energetics of the defect states and will lead to erroneous results.

For the structural relaxation, the calculated equilibrium lattice parameters and the unit cell volume  using DFT-GGA and DFT+U are listed in Table \ref{tab:hubbard_table}.  Both lattice parameter $a$ and $c$ increase with $U_d$ values regardless of $U_p$. However, the change is in order of 0.001 \AA \ which is relatively small and can be tolerated. Hence, DFT-U calculations do not substantially alter the lattice parameters. The c/a ratio determines the elongation of wurtzite ZnO along the c-axis. A decreasing trend of c/a ratio can be observed with increasing $U_d$. Ma et al. \citep{Ma2013} suggested that a strong on-site correction causes the localization of Zn 3d-states which further attracts the electron toward the core resulting to  compression of wurtzite ZnO. The calculated unit cell volume does not deviate by more than 1.5\%. The lattice parameters calculated using GGA are overestimate by 1.28\% and 2\% for lattice parameters $a$ and $c$, respectively. The calculated c/a ratio and unit cell volume are higher compared to the experimental values. However, the discrepancies in GGA calculations are considered to be tolerable and are still in good agreement with experimental data. Since the associated computational expense in structural relaxation, DFT+U relaxations will not be used. GGA will be used instead in structural relaxation of defect structures of ZnO  due to its computational simplicity and errors are generally tolerable.

An energy bandstructure diagram in Figure \ref{fig:hubbard_band-dos} shows the shifting of the conduction band upwards with the application of Hubbard-U correction. The blue dashed lines were calculated using DFT-GGA method while the solid red lines were calculated using DFT+U method.   The valence band consists of two separated regions, with widths of 5.61 eV and 2.70 eV, respectively. Standard GGA calculations resulted to the overlapping of these two regions  which are separated in DFT+U calculation. The conduction band minimum (CBM) is exactly similar in terms of width and dispersion for both calculations which differed only by a translation of energy due to widening of band gap. Also, the top of the valence band shows similar dispersion in both calculations but there are slight perturbations appearing 4 eV below the Fermi level. The observed similarities indicate that the effective masses of electrons and holes  are not affected by the DFT+U correction \citep{Goh2017}.

Analysis of the total density of states (DOS) and projected density of states (PDOS) was done to determine the modification of the bands due to  Hubbard correction on the atomic orbitals, as depicted in Figure \ref{fig:hubbard_pdos}. Figure \ref{fig:hubbard_pdos}a shows the DOS diagram of GGA and DFT+U. Clearly, the valence band consisted of two regions where the lower region is shifted to lower energies in DFT+U calculation, supported  by the bandstructure diagram in Figure \ref{fig:hubbard_band-dos}. The valence band is dominated by the Zn 3d states and O 2p states as indicated in Figure  \ref{fig:hubbard_pdos}b. However, the highest peak of Zn 3d orbital have shifted downwards while O 2p states moved by a negligible amount upon the application of DFT+U calculation as shown in Figure \ref{fig:hubbard_pdos}c. The DFT+U calculation reduces the artificial strong hybridization of O 2p states and Zn 3d states by lowering the position of the latter \citep{Goh2017}. The top of Zn 3d states is positioned 8 eV below the Fermi level. The obtained value was consistent with reports of Zwicker and Jacobi \citep{Zwicker1985} where they obtained a value of 7.8 eV by using angle-resolved photoelectron spectroscopy. X-ray photoemission results of Ruckh et al. \citep{Ruckh1994} and Vesely et al. \citep{Vesely1972} reported to have d-band position of 8.2 eV and 8.5 eV, respectively. The conduction band is composed mainly of O 2p states, Zn 3p and 4s states \citep{Harun2020}.



\begin{figure}[htbp!]
	\centering
	\includegraphics[width=0.7\linewidth]{"images/rnd/hubbard_val"}
	\caption[Variation of band gap with respect to Hubbard parameter $U_p$ applied to O 2p orbitals for a given $U_d$ applied to Zn 3d orbitals. ]{Variation of band gap with respect to Hubbard parameter $U_p$ applied to O 2p orbitals for a given $U_d$ values applied to Zn 3d orbitals. The horizontal dashed line indicates the experimental band gap of 3.37 eV.}
	\label{fig:hubbard_val}
\end{figure}

\begin{table}[htbp!]
	\centering
	\caption{Comparison of  structural parameters and electronic band gap using different DFT methods. }
	\resizebox{0.9\textwidth}{!}{%
		\begin{threeparttable}[t]
			\begin{tabular}{@{}cccccccc@{}}
				\toprule
				Method         & $U_d$ (eV) & $U_p$ (eV) & a (\AA) & c (\AA) & c/a    & Volume (\AA$^3$) & band gap (eV) \\ \midrule
				DFT+U          & 11         & 7.5        & 3.2458  & 5.2137  & 1.6063 & 47.5685          & 3.4085        \\
				DFT+U          & 13         & 7.0        & 3.2501  & 5.2197  & 1.6060 & 47.7503          & 3.3344        \\
				DFT+U          & 15         & 7.0        & 3.2517  & 5.2211  & 1.6057 & 47.8092          & 3.3950        \\
				DFT+U          & 17         & 6.5        & 3.2557  & 5.2272  & 1.6056 & 47.9827          & 3.2897        \\
				DFT+U          & 19         & 6.5        & 3.2569  & 5.2284  & 1.6053 & 48.0309          & 3.3312        \\
				GGA            & -          & -          & 3.2842  & 5.2982  & 1.6132 & 49.4886          & 0.7589        \\
				expt.\tnote{a} & -          & -          & 3.2427  & 5.1948  & 1.6020 & 47.3057          & 3.3700        \\ \bottomrule
			\end{tabular}
			% \vspace{-8pt}
			\begin{tablenotes}[para]
				\footnotesize
				\item[a] structural parameters were obtained from \citep{Sabine1969}, whereas the band gap value is from  \citep{Chitra2020}.
			\end{tablenotes}
		\end{threeparttable}
	}
	\label{tab:hubbard_table}
\end{table}

\begin{figure}[tbh!]
	\centering
	\includegraphics[width=0.7\linewidth]{"images/rnd/band-dos_juxtapose"}
	\caption[Energy bandstructures and total density of states obtained from standard DFT-GGA calculation (blue dashed lines) and DFT+U calculation (red solid lines)with $U_d =15$ and $U_p= 7$]{Energy bandstructures and total density of states obtained from standard DFT-GGA calculation (blue dashed lines) and DFT+U calculation (red solid lines) with $U_d =15$ and $U_p= 7$. Fermi level is shifted to zero as indicated by a horizontal dotted line. }
	\label{fig:hubbard_band-dos}
\end{figure}


\begin{figure}[tbh!]
	\centering
	\includegraphics[width=0.7\linewidth]{"images/rnd/dos-pdos_juxtapose"}
	\caption[The total density of states calculated using GGA (solid) and DFT+U (dashed) shown in (a) and the partial density of states of the constituent atoms using (b) GGA and (c) DFT+U]{ The total density of states calculated using GGA (solid) and DFT+U (dashed) shown in (a) and the partial density of states of the constituent atoms using (b) GGA and (c) DFT+U.}
	\label{fig:hubbard_pdos}
\end{figure}

\clearpage

\section{Defect Energetics}
As discussed earlier, the defect formation energies are calculated using Eq.  \eqref{eq:formation_E}. The calculated enthalpy of formation of bulk \ch{ZnO} is $\Delta H^f (\ch{ZnO}) = -348.02$ kJ/mol (-3.607 eV/unit cell) which is very close to the experimental value of $-350.46$ kJ/mol (-3.632 eV/unit cell) \citep{Medvedev1989}. The value is negative as expected for stable compounds. Table \ref{tab:E_formation} compares the formation energies of various native point defects in \ch{ZnO}. The defect formation energies are calculated under  \ch{Zn}-rich (\ch{O}-poor) and \ch{O}-rich (\ch{Zn}-poor) extreme conditions. Note that the formation energies depend on the chemical potentials and the position of the Fermi level $E_f$. The chemical potentials are in turn depended on the experimental conditions (i.e. \ch{O}-rich or \ch{Zn}-rich). For both zinc-rich and oxygen-rich environments, the oxygen vacancy with a $2+$ charge defect is expected to be stable for Fermi level positions near $E_f = 0$ (vbm). For Fermi level positions near the band gap, $E_g = 3.395$ eV, the neutral oxygen vacancy is stable for zinc-rich environment while zinc vacancy with a $2-$ charge defect is stable for oxygen-rich environment. Thus, the more negative the formation energy is, the more stable it is. Oxygen vacancies are likely to form for all positions of Fermi level except near the conduction band minimum in zinc-rich environment.

To know which specific position of Fermi level will have a transition of charge stabilities for a given type of defect, a thermodynamic transition levels was calculated using Eq. \eqref{eq:trans_level}. For instance, Figure \ref{fig:O_vac-form} shows the formation energies of oxygen vacancies for various charge states under two different conditions. For Fermi level positions less than 2.932 eV, the $2+$ is the most stable otherwise the neutral charge is the most stable. Thus, $E_f = 2.932$ eV is the transition level between the neutral and $2+$ state denoted as $\epsilon(0/2+)$. Note that 1+ charge state is unstable for any position of Fermi levels. Hence the transition levels $\epsilon(1+/2+)$ and $\epsilon(0/1+)$ are unstable since a much lower formation energies are available. In addition, the defect formation energies are shifted upward in \ch{O}-rich environment. This implies that oxygen vacancies are less likely to form on the said environment.

Figure \ref{fig:defect-formE} summarizes the formation energies and transition levels of the native point defects in ZnO. The figure only includes the charge defects since transition levels require at least two different charge states. Only these charge states that are energetically the most favorable at a given Fermi level are shown for each defect. The neutral defects such as \ch{O}$_i$, \ch{O}$_{\ch{Zn}}$, \ch{Zn}$_{\ch{O}}$ are excluded in the figure. The slopes of each segment correspond to the defect charge state. Kinks, or abrupt change in slopes, indicate the transition level between two different charge states. Defects that show positive slopes, such as the oxygen vacancy and zinc interstitial,  can have positively charged defects indicating their donor-type characteristics. On the other hand, zinc vacancy have negative slopes, a direct consequence of negative charge states, indicating that it has acceptor-type defects \citep{Oba2011}. In addition, the zinc vacancy $V_{\ch{Zn}}$ formation energy is shifted upwards in zinc-rich environment, which implies that the defect is not likely to form in that environment. In contrast, oxygen vacancy $V_{\ch{O}}$ and \ch{Zn}$_i$ are likely to form in oxygen-poor (zinc-rich) environments, thus, their formation energies are more negative. In general, defects that are associated with zinc excess or oxygen deficiency such as oxygen vacancy and zinc interstitial are donor-type defects. Those associated with zinc deficiency and oxygen excess, such as zinc vacancy, are acceptor-type defects.

Another way to visualize the transition levels is through plotting the Fermi level with the type of defect, as illustrated in Figure \ref{fig:transition_level}. The line segments correspond to open circles in Figure \ref{fig:defect-formE}. It can be observed that transition levels do not change in any chemical environments (i.e. Zn-rich or O-rich). This happens because the defect formation energies are shifted by the same amount  in any chemical environment such that taking their difference, as demonstrated in Eq. \eqref{eq:trans_level}, will leave the transition level invariant. Only the formation energies are changed, hence, shifting upwards or downwards are only observed in Figures \ref{fig:O_vac-form} and \ref{fig:defect-formE}. As observed in the Figure \ref{fig:transition_level}, the most positive charge state of a defect is the one that is stable at low Fermi levels. In other words, the most positive charge state for a given defect is likely to exist near the valence band. The most negative one, on the other hand, exists near the conduction band. A peculiar feature can be observed for the zinc interstitial. Its transition levels exceed the band gap of $E_g = 3.395$ eV. This means that it is possible to form resonance states within the conduction band. These states are located inside in the conduction band but are not actually occupied by an electrons since the electrons will migrate to lower energy state by transferring to the conduction band minimum \citep{Freysoldt2014}.

\begin{figure}[tbph!]
	\centering
	\includegraphics[width=0.65\linewidth]{"images/rnd/O_vac-formation"}
	\caption[Calculated defect formation energies of oxygen vacancy in Zn-rich and O-rich conditions]{Calculated defect formation energies of oxygen vacancy in Zn-rich and O-rich conditions. The zero of the Fermi level corresponds to the top of valence band and the dotted vertical line marks the position of the bottom of conduction band. The circles are the thermodynamic transition levels. Note that the $+$ charge state is unstable for any value of the Fermi level, hence, only $\epsilon(0/2+)$ is stable as indicated by the open circle. }
	\label{fig:O_vac-form}
\end{figure}

\begin{figure}[tbh!]
	\centering
	\includegraphics[width=0.65\linewidth]{"images/rnd/defect-formation"}
	\caption[Defect formation energies as a function of Fermi-level position for native point defects in ZnO]{Defect formation energies as a function of Fermi-level position for native point defects in ZnO. Results for Zn-rich and O-rich conditions are shown. The zero of Fermi level corresponds to the valence band maximum. The dotted vertical line corresponds to the conduction band minimum. The slope of these segments indicates the defect charge state. Abrupt change in slopes indicate transitions between different charge states.}
	\label{fig:defect-formE}
\end{figure}

\begin{figure}[tbh!]
	\centering
	\includegraphics[width=0.45\linewidth]{"images/rnd/trans_lvl"}
	\caption[Thermodynamic transition levels for native point defects in ZnO]{Thermodynamic transition levels for native point defects in ZnO. The horizontal dashed line is the band gap. VBM is set at zero Fermi level while CBM is set at the band gap.}
	\label{fig:transition_level}
\end{figure}

\begingroup
\setlength{\tabcolsep}{15pt}
\begin{table}[tbhp!]
	\centering
	\caption{Calculated defect formation energies $\Delta H^f$ at $E_f = 0$ and $E_f = E_g$ for native point defects in ZnO under zinc-rich and oxygen-rich conditions. Also shown in this table are the defect charge state $q$ and the number of constituent atoms, $n_{\ch{O}}$ and $n_{\ch{Zn}}$, added or removed in the defect supercell.}
	\label{tab:E_formation}
	\resizebox{0.9\textwidth}{!}{%
		\begin{tabular}{@{}cccccccccc@{}}
			\toprule
			\multirow{2}{*}{Defect}           & \multirow{2}{*}{$q$} & \multirow{2}{*}{$n_{\ch{O}}$} & \multirow{2}{*}{$n_{\ch{Zn}}$} &  & \multicolumn{2}{c}{$\Delta H^f$ (Zn-rich) (eV)} &             & \multicolumn{2}{c}{ $\Delta H^f$ (O-rich) (eV)}                           \\ \cmidrule(lr){6-7} \cmidrule(l){9-10}
			                                  &                      &                               &                                &  & $E_f = 0$                                       & $E_f = E_g$ &                                                 & $E_f = 0$ & $E_f = E_g$ \\ \midrule
			$V_{\ch{O}}$                      & 0                    & -1                            & 0                              &  & -0.018                                          & -0.018      &                                                 & 3.589     & 3.589       \\
			                                  & +1                   & -1                            & 0                              &  & -2.950                                          & 0.440       &                                                 & 0.657     & 4.047       \\
			                                  & +2                   & -1                            & 0                              &  & -6.466                                          & 0.314       &                                                 & -2.859    & 3.921       \\
			\multirow[t]{3}{*}{$V_{\ch{Zn}}$} & 0                    & 0                             & -1                             &  & 9.821                                           & 9.821       &                                                 & 6.214     & 6.214       \\
			                                  & -1                   & 0                             & -1                             &  & 10.474                                          & 7.084       &                                                 & 6.867     & 3.477       \\
			                                  & -2                   & 0                             & -1                             &  & 12.004                                          & 5.224       &                                                 & 8.396     & 1.616       \\
			\ch{O}$_i$                        & 0                    & 1                             & 0                              &  & 11.558                                          & 11.558      &                                                 & 7.951     & 7.951       \\
			                                  & -1                   & 1                             & 0                              &  & 12.988                                          & 9.598       &                                                 & 9.381     & 5.991       \\
			                                  & -2                   & 1                             & 0                              &  & 15.050                                          & 8.270       &                                                 & 11.443    & 4.663       \\
			\multirow[t]{3}{*}{\ch{Zn}$_i$}   & 0                    & 0                             & 1                              &  & 4.277                                           & 4.277       &                                                 & 7.884     & 7.884       \\
			                                  & +1                   & 0                             & 1                              &  & -0.569                                          & 2.821       &                                                 & 3.038     & 6.428       \\
			                                  & +2                   & 0                             & 1                              &  & -4.481                                          & 2.299       &                                                 & -0.874    & 5.906       \\
			\ch{O}$_{\ch{Zn}}$                & 0                    & 1                             & -1                             &  & 15.495                                          & 15.495      &                                                 & 8.281     & 8.281       \\
			\ch{Zn}$_{\ch{O}}$                & 0                    & -1                            & 1                              &  & 3.561                                           & 3.561       &                                                 & 10.775    & 10.775      \\ \bottomrule
		\end{tabular}%
	}
\end{table}
\endgroup

\clearpage

\section{Oxygen Vacancies}
Based on the data in Table \ref{tab:E_formation}, oxygen vacancies have the lowest formation energies among the native donor-type defects. Since the $1+$ charge state is unstable for all positions in the Fermi level, only one transition level can be observed as supported in Figure \ref{fig:transition_level}. This was experimentally verified by Hofmann et al. using deep level transient spectroscopy (DLTS) \citep{Hofmann2007}.
The stable transition level $\epsilon(0/2+)$ is located around 0.171 eV below the conduction band. However, the position of this level depends on the DFT approach used as some published results ranges from 0.67 eV to 2.5 eV below the conduction band \citep{Lany2010, Oba2008, Paudel2008}. The underestimation might be due to DFT+U used in this study since the Hubbard $U$ parameter used is not suitable for the calculation of atomic chemical potential for the metallic bulk phase of \ch{Zn} \citep{Lany2010}. However, qualitatively speaking, oxygen vacancy is considered to be a deep donor since the energy required to add electron to the conduction band is much higher than the thermal energy of $k_B T = $ 0.0259 eV at room temperature \citep{Freysoldt2014}. Oxygen vacancies are usually associated as an origin of unintentional $n$-type conductivity (experimental $E_f$ near the conduction band) in ZnO. However, oxygen vacancy is a deep donor, thus, it cannot donate electrons to the conduction band by mere thermal excitation. Therefore, it cannot be a source of $n$-type conductivity. Note that the formation energy of $V_{\ch{O}}$ is relatively higher in  $n$-type ZnO  but much lower in $p$-type ZnO where it exists as $2+$ charge state. However, for the \ch{Zn}-rich environment, the formation energy of neutral $V_{\ch{O}}$  near the conduction band is -0.018 eV which is considered stable. Thus, it accounts for the observed non-stoichiometry of ZnO  \citep{Oba2011}. On the other hand, for the \ch{O}-rich environment, the formation energy of neutral $V_{\ch{O}}$  near the conduction band is quite high in $n$-type material ($\approx$ 3.589 eV). This implies that  $V_{\ch{O}}$ concentrations are low enough in equilibrium conditions that it cannot provide electrons to the conduction band and cannot serve as donor level in this environment. For both environments, oxygen vacancies can form in $p$-doped ZnO serving as a potential source of compensation for holes \citep{Janotti2007}.


The neutral \ch{O} vacancy induces a deep and localized one-electron (Kohn-Sham) state in the band gap as shown in Figure \ref{fig:band-O_vac}a. The state is located 1.34 eV above the valence band maximum (2.055 eV below the band gap). Note that Kohn-Sham states do not correspond directly to the thermodynamic transition levels since the latter is calculated in the basis of Eq. \eqref{eq:formation_E}. The partial density of states shown in Figure \ref{fig:dos-O_vac}a verified that this state is attributed to the 2p orbital of oxygen. For the $1+$ charge defect state, the localized electronic state is shifted upwards with 2.35 eV above the valence band as shown in Figure  \ref{fig:band-O_vac}b  while the $2+$ charge state has a electronic state located 3.10 eV above the valence band as shown in Figure \ref{fig:band-O_vac}c. Note that the electronic state of $2+$ charge defect follows the dispersion of the conduction band particularly near the $\Gamma$ point. Thus this state is not localized anymore in $2+$ charge defect state of \ch{O} vacancy. This is supported by the broadening of the defect peak observed in the partial density of states in Figure \ref{fig:dos-O_vac}c in comparison to a narrow localized peaks of neutral and $1+$ charge state. Both electronic states in the defect charge  states $1+$ and $2+$ are attributed to O 2p orbitals as shown in Figure \ref{fig:dos-O_vac}b and \ref{fig:dos-O_vac}c, respectively.

\begin{figure}[tbph!]
	\centering
	\includegraphics[width=0.93\textwidth]{"images/rnd/O_vac-band"}
	\caption[Band structure of Oxygen vacancy with its possible defect charge states]{Band structure of Oxygen vacancy with its possible defect charge states. }
	\label{fig:band-O_vac}
\end{figure}

\begin{figure}[tbph!]
	\centering
	\includegraphics[width=0.93\textwidth]{"images/rnd/O_vac-dos"}
	\caption[Total density of states (TDOS) and partial density of states (PDOS) of Oxygen vacancy in different  defect charge states]{Total density of states (TDOS) and partial density of states (PDOS) of Oxygen vacancy in different  defect charge states.}
	\label{fig:dos-O_vac}
\end{figure}


\section{Zinc Vacancies}
The Zn vacancy has been proposed to be a dominant acceptor-type defect in ZnO as verified by the existence of negative charge defect states in Figure \ref{fig:defect-formE} and the location of the electronic states in Figure \ref{fig:band-Zn_vac}a. The \ch{Zn} vacancy has two defect transition levels located at
$\epsilon(0/-)=$ 0.653 eV and  $\epsilon(-/2-)=$ 1.530 eV, respectively. It can be concluded that $V_{\ch{Zn}}$  is a deep acceptor since it has higher energy to remove the electron from the valence band than the thermal energy at room temperature.

With increasing Fermi level, zinc vacancies are more stable  in $n$-type ZnO. In addition, they are also more favorable in oxygen-rich conditions since it is associated with zinc deficiency. In the $p$-type case, the $V_{\ch{Zn}}$ has an unusually high formation energies as illustrated in Figure \ref{fig:defect-formE}. Therefore,  Zn vacancy is unlikely to form at a substantial concentration, suggesting that it cannot act as an acceptor in any chemical conditions. However, due to its stability in $n$-type ZnO, zinc vacancies can act as a compensation center for which it exists in $2-$ charge state. The presence of this defect in $n$-type material has been verified by positron annihilation spectroscopy experiments \citep{Tuomisto2003,Tuomisto2005}.

Figure \ref{fig:band-Zn_vac}a shows a very narrow peak near the valence band in the band structure of neutral \ch{Zn} vacancy. Upon closer inspection, the defect state consisted of two electronic bands, as shown in Figure \ref{fig:bandclose-Zn_vac}a, which are both attributed to O 2p orbitals as verified by the partial density of states in Figure \ref{fig:dos-Zn_vac}a. Note that some parts of bands are below the valence band minimum, implying that electrons can be occupied in this region. Similarly, the charge defects exhibit the same bandstructure as the neutral one, as illustrated by Figures \ref{fig:band-Zn_vac}b and \ref{fig:band-Zn_vac}c for the charge defect state of $1-$ and $2-$.  For $1-$ charge state, the electronic defect state shifts upward such that no parts of the bands are below the Fermi level as illustrated in Figure \ref{fig:bandclose-Zn_vac}b. However, in Figure \ref{fig:bandclose-Zn_vac}c, three separate bands are observed in $2-$ charge state, both are above the valence band minimum. All the electronic defects are associated with 2p orbitals as illustrated in \ref{fig:dos-Zn_vac}b and \ref{fig:dos-Zn_vac}c.

\begin{figure}[tbph!]
	\centering
	\includegraphics[width=1\textwidth]{"images/rnd/Zn_vac-band"}
	\caption[Band structure of Zinc vacancy with its possible defect charge states]{Band structure of Zinc vacancy with its possible defect charge states. }
	\label{fig:band-Zn_vac}
\end{figure}

\begin{figure}[tbph!]
	\centering
	\includegraphics[width=1\textwidth]{"images/rnd/Zn_vac-bandclose"}
	\caption[Band structure of Zinc vacancy with its possible defect charge states near the valence band]{Band structure of Zinc vacancy with its possible defect charge states near the valence band. }
	\label{fig:bandclose-Zn_vac}
\end{figure}

\clearpage

\begin{figure}[t!]
	\centering
	\includegraphics[width=1\textwidth]{"images/rnd/Zn_vac-dos"}
	\caption[Total density of states (TDOS) and partial density of states (PDOS) of Zinc vacancy in different  defect charge states]{Total density of states (TDOS) and partial density of states (PDOS) of Zinc vacancy in different  defect charge states.}
	\label{fig:dos-Zn_vac}
\end{figure}


\section{Oxygen Interstitials}
Oxygen interstitials happen when a material is exposed in excess oxygen or deficient zinc environment. The interstitial oxygen atom can occupy either at the tetrahedral or the octahedral site. It was reported that interstitial at tetrahedral site is unstable and will spontaneously relax to octahedral configuration \citep{Erhart2005}. Oxygen interstitials are expected to behave as an acceptor defect since the interstitial oxygen atom can accommodate electrons from the valence band. This is supported by the bandstructure calculation shown in Figure \ref{fig:band-O_i}a where sharp peaks are observed near the valence band.  These sharp peaks are attributed to the defect levels upon  closer inspection of the bandstructure shown in Figure \ref{fig:bandclose-O_i}a. Three defect band states can be observed. The top two bands form a localized state while the bottom band has a higher dispersion and mixes with the valence band in the $\Gamma-A$ path. Hence, this defect state form resonant states. This resonant state is occupied by electrons since it is below the Fermi level. All the defect states are derived from oxygen $p$ orbitals as verified in Figure \ref{fig:dos-O_i}a.

Oxygen interstitials have very high defect formation energies ($>$ 7 eV). Hence, this defect are not expected to play important roles under thermal equilibrium conditions.

\begin{figure}[tbph!]
	\centering
	\includegraphics[width=1\textwidth]{"images/rnd/O_i-band"}
	\caption[Band structure of Oxygen interstitial with its possible defect charge states]{Band structure of Oxygen interstitial with its possible defect charge states. }
	\label{fig:band-O_i}
\end{figure}

\begin{figure}[tbph!]
	\centering
	\includegraphics[width=1\textwidth]{"images/rnd/O_i-bandclose"}
	\caption[Band structure of Oxygen interstitial with its possible defect charge states near the valence band]{Band structure of Oxygen interstitial with its possible defect charge states near the valence band. }
	\label{fig:bandclose-O_i}
\end{figure}

\begin{figure}[t!]
	\centering
	\includegraphics[width=1\textwidth]{"images/rnd/O_i-dos"}
	\caption[Total density of states (TDOS) and partial density of states (PDOS) of Oxygen interstitial in different  defect charge states]{Total density of states (TDOS) and partial density of states (PDOS) of Oxygen interstitial in different  defect charge states.}
	\label{fig:dos-O_i}
\end{figure}

\clearpage

\section{Zinc Interstitials}
% Zinc 

For Zn interstitial, the octahedral site was considered in this study. The tetrahedral site is reported to be energetically less favorable or unstable \citep{Oba2001,Zhao2006,Oba2010}. With respect to formation energies, it has a high value   $\sim$ 4.277 eV under the $n$-type conditions at the  Zn-rich limit and a much higher value  $\sim$ 7.884 eV at O-rich limit. Hence, the concentration of Zn interstitial is expected to be low in $n$-type ZnO for any chemical environments. Zn interstitials are unlikely to be responsible for unintentional $n$-type conductivity observed in ZnO. For the $p$-type case, Zn interstitial has a negative defect formation energy at both Zn-rich and O-rich limit, implying that it is stable and can readily form. The excess Zn atom in the interstitial will donate electrons, thus this defect is expected to be a donor type and can serve as a a potential source of compensation in $p$-type ZnO \citep{Janotti2007}.

The calculated transition levels $\epsilon(2+/+)$ and $\epsilon(+/0)$ are all above the conduction band minimum as shown in Figure \ref{fig:transition_level}. This agrees well with the results of Janotti et al. \citep{Janotti2007} where zinc interstitials occur exclusively as a $2+$ charge defect in all Fermi levels in the band gap. Hence, zinc interstitial will always donate electrons to the conduction band since it exists solely as a $2+$ charge defect. This defect is also a shallow donor since it forms a  defect state near or resonant state inside the conduction band as experimentally verified in the high-energy electron irradiation measurements of Look et al. \citep{Look1999}. The electrons in the resonant state will 	migrate to lower energy state by transferring to the conduction band minimum \citep{Freysoldt2014}.

The existence of defect state near the conduction band as observed in Figure \ref{fig:band-dos_Zn_i} verifies that Zn interstitial is a shallow donor. This defect state is 0.41 eV below the conduction band minimum.  The defect state is delocalized especially in regions near the $\Gamma$ point where it follows a free electron-like dispersion. Due to this delocalized nature, the  removal of electrons does not alter the bandstructure. Hence, similar band structures have been observed for the $+$ and $2+$ charge states as shown in Figure
\ref{fig:band-dos_Zn_i-p1} and \ref{fig:band-dos_Zn_i-p2}, respectively. The defect state always exists as $2+$ charge defect and the two donor electrons are located in the conduction band minimum. The possibility of the formation of resonant state cannot be guarantee by bandstructure calculations since this state is mixed in the conduction band, thus it is difficult to locate. For the neutral and charged zinc interstitials, the defect state is composed of O-2s, O-2p, Zn-4s, and Zn-3p states as shown by the Figures \ref{fig:dos-pdos_Zn_i}, \ref{fig:dos-pdos_Zn_i-p1}, and \ref{fig:dos-pdos_Zn_i-p2} for neutral, $1+$, and $2+$ charge defects, respectively.

\begin{figure}[tbh!]
	\centering
	\includegraphics[width=0.6\linewidth]{"images/rnd/band-dos_Zn_i"}
	\caption[Bandstructure of Zinc interstitial]{Bandstructure of Zinc interstitial}
	\label{fig:band-dos_Zn_i}
\end{figure}

\begin{figure}[tbh!]
	\centering
	\includegraphics[width=0.6\linewidth]{"images/rnd/dos-pdos_Zn_i"}
	\caption[Partial density of states of Zinc interstitial]{Partial density of states of Zinc interstitial}
	\label{fig:dos-pdos_Zn_i}
\end{figure}

% \begin{figure}[tbh!]
% 	\centering
% 	\includegraphics[width=0.5\linewidth]{"images/rnd/dos-pdos-zoom_Zn_i"}
% 	\caption[Partial density of states of Zinc interstitial near Conduction Band]{Partial density of states of Zinc interstitial near Conduction Band}
% 	%\label{fig:free-electron}
% \end{figure}

% +1 Zinc interstitial 
\begin{figure}[tbh!]
	\centering
	\includegraphics[width=0.6\linewidth]{"images/rnd/band-dos_Zn_i-p1"}
	\caption[Bandstructure of +1 Zinc interstitial]{Bandstructure of +1 Zinc interstitial}
	\label{fig:band-dos_Zn_i-p1}
\end{figure}

\begin{figure}[tbh!]
	\centering
	\includegraphics[width=0.6\linewidth]{"images/rnd/dos-pdos_Zn_i-p1"}
	\caption[Partial density of states of +1 Zinc interstitial]{Partial density of states of +1 Zinc interstitial }
	\label{fig:dos-pdos_Zn_i-p1}
\end{figure}

% +2 Zinc interstitial 

\begin{figure}[tbh!]
	\centering
	\includegraphics[width=0.6\linewidth]{"images/rnd/band-dos_Zn_i-p2"}
	\caption[Bandstructure of +2 Zinc interstitial]{Bandstructure of +2 Zinc interstitial}
	\label{fig:band-dos_Zn_i-p2}
\end{figure}

\begin{figure}[tbh!]
	\centering
	\includegraphics[width=0.6\linewidth]{"images/rnd/dos-pdos_Zn_i-p2"}
	\caption[Partial density of states of +2 Zinc interstitial]{Partial density of states of +2 Zinc interstitial }
	\label{fig:dos-pdos_Zn_i-p2}
\end{figure}


\clearpage

\section{Oxygen Antisites}
Oxygen antisites \ch{O}$_{\ch{Zn}}$ have the highest defect formation energies with a value of 15.495 eV in the zinc-rich environment. It also the second highest formation energy in the oxygen-rich environment with a value of 8.281 eV (see Table \ref{tab:E_formation}). Thus, oxygen antisites are not expected to form in thermal equilibrium conditions. However, it could form in nonequilibrium conditions such as under irradiation or ion implantation. Since a zinc atom on the ZnO lattice is replaced by an oxygen atom for which it accepts electrons, oxygen antisites are expected to have acceptor-like defects. This is supported by the presence of a defect state near the valence band as shown in Figure \ref{fig:band-dos_O_anti}.  However, there are also other defect states observed near the midgap. Upon closer inspection, these states are centered at 2.60 eV above the top of valence band, as shown in Figure \ref{fig:band-dos-close_O_anti}. The shallow acceptor defect is located 0.122 eV above the top of valence band when measured at the $\Gamma$ point. If optical transitions are possible between these defects, then an electron trapped in the localized defect near the midgap and relaxes to the shallow acceptor defect will emit light with an energy of 2.478 eV  (2.60 eV - 0.122 eV) which falls inside the range of green light. Thus, oxygen antisites might cause the experimentally observed  green luminescence in ZnO \citep{Djurisic2007,Empizo2014,,Cizek2015}. Lin et al. \citep{Lin2001} have argued that the green luminescence in ZnO corresponds to deep levels induced by oxygen antisites. However, it was shown in this study that oxygen antisites are not stable due to its high defect formation energies, as supported by the works of Janotti et al. \citep{Janotti2007}. There are still no general consensus on the correct assignment of the green luminescence, since different authors have assign various defects as the origin of green luminescence \citep{Reynolds1997,Oezguer2005,Djurisic2007}. The origin of the defect states near the midgap are not clear yet since the possible defects that can form out of oxygen antisites is either oxygen interstitial or zinc vacancy but both defects have acceptor-like properties. However, it can be hypothesized that the defect state near the midgap resembles the deep defect states of oxygen interstitial as shown in Figure \ref{fig:band-dos-close_O_i} while the shallow acceptor resembles the defect states of zinc vacancy as shown in Figure \ref{fig:band-dos-close_Zn_vac}. Nevertheless, all the defects are attributed to the O-2p orbital as  verified in Figure \ref{fig:dos-pdos_O_anti}.


\begin{figure}[tbh!]
	\centering
	\includegraphics[width=0.6\linewidth]{"images/rnd/band-dos_O_anti"}
	\caption[Bandstructure of Oxygen antisite]{Bandstructure of Oxygen antisite}
	\label{fig:band-dos_O_anti}
\end{figure}

\begin{figure}[tbh!]
	\centering
	\includegraphics[width=0.6\linewidth]{"images/rnd/dos-pdos_O_anti"}
	\caption[Partial density of states of Oxygen antisite]{Partial density of states of Oxygen antisite}
	\label{fig:dos-pdos_O_anti}
\end{figure}

\begin{figure}[tbh!]
	\centering
	\includegraphics[width=0.6\linewidth]{"images/rnd/band-dos-close_O_anti"}
	\caption[Bandstructure of Oxygen antisite in the band gap region]{Bandstructure of Oxygen antisite  in the band gap region}
	\label{fig:band-dos-close_O_anti}
\end{figure}
\clearpage

\section{Zinc Antisites}
Zinc antisites \ch{Zn}$_{\ch{O}}$ are form when a zinc atom substitutes one of the oxygen atom in the ZnO lattice.
Zinc antisites have relatively high defect formation energies, thus they are unlikely to occur under equilibrium conditions in ZnO for any position of the Fermi level in the band gap. Zinc antisites are expected to be a donor since the  zinc has two valence electrons that it can donate to the conduction band.

Zinc antisite can be thought as a complex of oxygen vacancy and zinc interstitial. This is reasonable since an oxygen atom must be removed first before adding the zinc atom on the same location.  Thus in the bandstructure diagram shown in Figure \ref{fig:bands.Zn-anti}, two defects can be observed. One is in the lower part of the band gap that resembles the defect state of oxygen vacancy, and the other is a state resonant with the conduction band that resembles the zinc interstitial state \citep{Janotti2007}. The existence of the resonant state can be shown by a DOS plot in Figure \ref{fig:dos.Zn-anti} where a notable peak is observed in the bottom part of the conduction band. The electrons in the resonant state can occupy the conduction band minimum that causes the zinc antisite to act as a shallow donor. Both defects are dominated by the O-2p orbitals as shown by the partial density of states diagram in \ref{fig:dos.Zn-anti}.

\begin{figure}[tbh!]
	\centering
	\includegraphics[width=0.6\linewidth]{"images/rnd/band-dos_Zn_anti"}
	\caption[Bandstructure of Zinc antisite]{Bandstructure of Zinc antisite}
	\label{fig:bands.Zn-anti}
\end{figure}

\begin{figure}[tbh!]
	\centering
	\includegraphics[width=0.6\linewidth]{"images/rnd/dos-pdos_Zn_anti"}
	\caption[Partial density of states of Zinc antisite]{Partial density of states of Zinc antisite}
	\label{fig:dos.Zn-anti}
\end{figure}


