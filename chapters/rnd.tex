\chapter{Results and Discussion} \label{chap:rnd}
\section{Convergence Tests}
Convergence tests on the ZnO unit cell were done by varying the k-points, and the kinetic energy cut-offs  of the Kohn-Sham orbital and the electron density, respectively.  As shown in Figure \ref{fig:conv_k}, the calculated pressure and total energy converges at $k \geq 2$ which corresponds to the Monkhorst-Pack grid of $7 \times 7 \times 2$. One can choose this minimum k-point required for convergence to be used for subsequent calculations. However, higher values are preferred as to make sure the convergence is guaranteed for all configuration of the system. In this thesis, the k-point used was set $k = 6$ which corresponds to the Monkhorst-Pack grid of $11 \times 11 \times 6$. Next, the cutoff energy of the Kohn-Sham orbitals $E_{\text{wfc}}$ were varied until convergence was achieved. At $E_{\text{wfc}} = 40$ Ry, very small fluctuations are observed for both the total energy and pressure as verified in Figure \ref{fig:conv_wfc}. However, $E_{\text{wfc}} = 90$ Ry was chosen so that it can take into account the possible errors due to supercell image interaction and spurious interaction of defects \citep{White1994,Makov1995,Martyna1999}. Lastly, the cutoff energy of the electron density $E_{\text{rho}}$ converged at 720 Ry corresponding to the dual value of 8 ($8 \times E_{\text{wfc}}$) as illustrated in Figure \ref{fig:conv_rho}. The values of $E_{\text{wfc}} = 90$ Ry, $E_{\text{rho}} = 720$ Ry, and Monkhorst-Pack of $11 \times 11 \times 6$ will be used as convergence parameters for all supercell simulations. 

\begin{figure}[tbh!] 
	\centering
	\includegraphics[width=0.7\linewidth]{"images/rnd/kpoints"}
	\caption[Convergence of pressure and total energies with respect to k-points]{Convergence of pressure and total energies with respect to k-points. The x-axis denotes the k-point $k$ in the Monkhorst-Pack grid of $(k+5) \times (k +5 ) \times k$}
	\label{fig:conv_k}
\end{figure}

\begin{figure}[tbh!]
	\centering
	\includegraphics[width=0.7\linewidth]{"images/rnd/ecutwfc"}
	\caption[Convergence of pressure and total energies with respect to cutoff energy of the Kohn-Sham orbital]{Convergence of pressure and total energies with respect to cutoff energy of the Kohn-Sham orbital $E_{\text{wfc}}$.}
	\label{fig:conv_wfc}
\end{figure}

\begin{figure}[tbh!]
	\centering
	\includegraphics[width=0.7\linewidth]{"images/rnd/ecutrho"}
	\caption[Convergence of pressure and total energies with respect to cutoff energy of the electron density]{Convergence of pressure and total energies with respect to cutoff energy of the electron density $E_{\text{rho}}$.}
	\label{fig:conv_rho}
\end{figure}

\section{Hubbard-U parameters}
The influence of both $U_p$, acting on O 2p orbitals,  and $U_d$, acting on Zn 3d orbitals, on the p-d hybridization is evident in the variation of band gap as shown in Figure \ref{fig:hubbard_val}. For a fixed $U_d$, there is a huge increase in the magnitude of the electronic band gap of almost 0.2 eV with the increment of $U_p$ by 0.5 eV. The rate of increase in band gap is high enough that most band gap values have exceeded the reference band gap, shown as a horizontal dashed line in Figure \ref{fig:hubbard_val}.  On the other hand,  the band gap increases steadily with $U_d$ for a given $U_p$. The higher the $U_d$ value, the lower $U_p$ needed to widen the band gap. In either case, the band gap monotonically increase with both parameters. The observed trend agrees very well with the recent results of Goh et al. \citep{Goh2017}. 


Among the values, $U_d = 15$ eV and $U_p = 7$ eV has the band gap  value of 3.395 eV which is closest to the experimental value of 3.37 eV. Other combinations of $U_d$ and $U_p$ listed in Table \ref{tab:hubbard_table} are equally valid in correcting the band gap. Nevertheless, this study chose $U_d = 15$ eV and $U_p = 7$ eV as the correction parameters since the calculated band gap falls between the experimental band gap of 3.37 eV to 3.44 eV. Standard DFT calculations using GGA will produce severe underestimation of the band gap with  a value of 0.7589 eV, an error of more than 75\%, consistent with the reports of Haq et al. \citep{Haq2013} and Erhart et al. \citep{Erhart2006a}. This underestimation will affect energetics of the defect states and will lead to erroneous results.

For the structural relaxation, the calculated equilibrium lattice parameters and the unit cell volume  using DFT-GGA and DFT+U are listed in Table \ref{tab:hubbard_table}.  Both lattice parameter $a$ and $c$ increase with $U_d$ values regardless of $U_p$. However, the change is in order of 0.001 \AA \ which is relatively small and can be tolerated. Hence, DFT-U calculations do not substantially alter the lattice parameters. The c/a ratio determines the elongation of wurtzite ZnO along the c-axis. A decreasing trend of c/a ratio can be observed with increasing $U_d$. Ma et al. \citep{Ma2013} suggested that a strong on-site correction causes the localization of Zn 3d-states which further attracts the electron toward the core resulting to  compression of wurtzite ZnO. The calculated unit cell volume does not deviate by more than 1.5\%. The lattice parameters calculated using GGA are overestimate by 1.28\% and 2\% for lattice parameters $a$ and $c$, respectively. The calculated c/a ratio and unit cell volume are higher compared to the experimental values. However, the discrepancies in GGA calculations are considered to be tolerable and are still in good agreement with experimental data. Since the associated computational expense in structural relaxation, DFT+U relaxations will not be used. GGA will be used instead in structural relaxation of defect structures of ZnO  due to its computational simplicity and errors are generally tolerable.

An energy bandstructure diagram in Figure \ref{fig:hubbard_band-dos} shows the shifting of the conduction band upwards with the application of Hubbard-U correction. The blue dashed lines were calculated using DFT-GGA method while the solid red lines were calculated using DFT+U method.   The valence band consists of two separated regions, with widths of 5.61 eV and 2.70 eV, respectively. Standard GGA calculations resulted to the overlapping of these two regions  which are separated in DFT+U calculation. The conduction band minimum (CBM) is exactly similar in terms of width and dispersion for both calculations which differed only by a translation of energy due to widening of band gap. Also, the top of the valence band shows similar dispersion in both calculations but there are slight perturbations appearing 4 eV below the Fermi level. The observed similarities indicate that the effective masses of electrons and holes  are not affected by the DFT+U correction \citep{Goh2017}. 

Analysis of the total density of states (DOS) and projected density of states (PDOS) was done to determine the modification of the bands due to  Hubbard correction on the atomic orbitals, as depicted in Figure \ref{fig:hubbard_pdos}. Figure \ref{fig:hubbard_pdos}a shows the DOS diagram of GGA and DFT+U. Clearly, the valence band consisted of two regions where the lower region is shifted to lower energies in DFT+U calculation, supported  by the bandstructure diagram in Figure \ref{fig:hubbard_band-dos}. The valence band is dominated by the Zn 3d states and O 2p states as indicated in Figure  \ref{fig:hubbard_pdos}b. However, the highest peak of Zn 3d orbital have shifted downwards while O 2p states moved by a negligible amount upon the application of DFT+U calculation as shown in Figure \ref{fig:hubbard_pdos}c. The DFT+U calculation reduces the artificial strong hybridization of O 2p states and Zn 3d states by lowering the position of the latter \citep{Goh2017}. The top of Zn 3d states is positioned 8 eV below the Fermi level. The obtained value was consistent with reports of Zwicker and Jacobi \citep{Zwicker1985} where they obtained a value of 7.8 eV by using angle-resolved photoelectron spectroscopy. X-ray photoemission results of Ruckh et al. \citep{Ruckh1994} and Vesely et al. \citep{Vesely1972} reported to have d-band position of 8.2 eV and 8.5 eV, respectively. The conduction band is composed mainly of O 2p states, Zn 3p and 4s states \citep{Harun2020}.



\begin{figure}[htbp!]
	\centering
	\includegraphics[width=0.7\linewidth]{"images/rnd/hubbard_val"}
	\caption[Variation of band gap with respect to Hubbard parameter $U_p$ applied to O 2p orbitals for a given $U_d$ applied to Zn 3d orbitals. ]{Variation of band gap with respect to Hubbard parameter $U_p$ applied to O 2p orbitals for a given $U_d$ values applied to Zn 3d orbitals. The horizontal dashed line indicates the experimental band gap of 3.37 eV.}
	\label{fig:hubbard_val}
\end{figure}

\begin{table}[htbp!]
	\centering
	\caption{Comparison of  structural parameters and electronic band gap using different DFT methods. }
	\resizebox{0.9\textwidth}{!}{%
	\begin{threeparttable}[t]
	\begin{tabular}{@{}cccccccc@{}}
	\toprule
	Method & $U_d$ (eV) & $U_p$ (eV)  & a (\AA)     & c (\AA)     & c/a    & Volume (\AA$^3$) & band gap (eV) \\ \midrule
	DFT+U  & 11 & 7.5 & 3.2458 & 5.2137 & 1.6063 & 47.5685 & 3.4085   \\
	DFT+U  & 13 & 7.0 & 3.2501 & 5.2197 & 1.6060 & 47.7503 & 3.3344   \\
	DFT+U  & 15 & 7.0 & 3.2517 & 5.2211 & 1.6057 & 47.8092 & 3.3950    \\
	DFT+U  & 17 & 6.5 & 3.2557 & 5.2272 & 1.6056 & 47.9827 & 3.2897   \\
	DFT+U  & 19 & 6.5 & 3.2569 & 5.2284 & 1.6053 & 48.0309 & 3.3312   \\
	GGA    &  -  &  -   & 3.2842 & 5.2982 & 1.6132 & 49.4886 & 0.7589   \\
	expt.\tnote{a} &  -  &   -  & 3.2427 & 5.1948 & 1.6020 & 47.3057 & 3.3700     \\ \bottomrule
	\end{tabular}
	\vspace{-8pt}
	\begin{tablenotes}[para]
		\footnotesize
		\item[a] structural parameters were obtained from \citep{Sabine1969}, whereas the band gap value is from  \citep{Chitra2020}.
	\end{tablenotes}
	\end{threeparttable}
	}
	\label{tab:hubbard_table}
	\end{table}

	\begin{figure}[tbh!]
		\centering
		\includegraphics[width=0.7\linewidth]{"images/rnd/band-dos_juxtapose"}
		\caption[Energy bandstructures and total density of states obtained from standard DFT-GGA calculation (blue dashed lines) and DFT+U calculation (red solid lines)with $U_d =15$ and $U_p= 7$]{Energy bandstructures and total density of states obtained from standard DFT-GGA calculation (blue dashed lines) and DFT+U calculation (red solid lines) with $U_d =15$ and $U_p= 7$. Fermi level is shifted to zero indicated as horizontal dotted line. }
		\label{fig:hubbard_band-dos}
	\end{figure}


	\begin{figure}[tbh!]
		\centering
		\includegraphics[width=0.7\linewidth]{"images/rnd/dos-pdos_juxtapose"}
		\caption[The total density of states calculated using GGA (solid) and DFT+U (dashed) shown in (a) and the partial density of states of the constituent atoms using (b) GGA and (c) DFT+U]{ The total density of states calculated using GGA (solid) and DFT+U (dashed) shown in (a) and the partial density of states of the constituent atoms using (b) GGA and (c) DFT+U.}
		\label{fig:hubbard_pdos}
	\end{figure}

\clearpage

\section{Defect Energetics}
As discussed earlier, the defect formation energy is calculated using Eq.  \eqref{eq:formation_E}. 


Oxygen vacancies are likely to form for all positions of Fermi level. 

The calculated enthalpy of formation of \ch{ZnO}
\begin{figure}[tbh!]
	\centering
	\includegraphics[width=0.65\linewidth]{"images/rnd/O_vac-formation"}
	\caption[Calculated defect formation energies of oxygen vacancy in Zn-rich and O-rich conditions]{Calculated defect formation energies of oxygen vacancy in Zn-rich and O-rich conditions. The zero of the Fermi level corresponds to the top of valence band and the dotted vertical line marks the position of the bottom of conduction band. The circles are the thermodynamic transition levels. Note that the $+$ charge state is unstable for any value of the Fermi level, hence, only $\epsilon(0/2+)$ is stable as indicated by the open circle. }
	%\label{fig:free-electron}
\end{figure}

\begin{figure}[tbh!]
	\centering
	\includegraphics[width=0.65\linewidth]{"images/rnd/defect-formation"}
	\caption[Defect formation energies as a function of Fermi-level position for native point defects in ZnO]{Defect formation energies as a function of Fermi-level position for native point defects in ZnO. Results for Zn-rich and O-rich conditions are shown. The zero of Fermi level corresponds to the valence band maximum. The slope of these segments indicates the defect charge state. Abrupt change in slopes indicate transitions between different charge states.}
	%\label{fig:free-electron}
\end{figure}

% \begingroup
% \setlength{\tabcolsep}{20pt}
% \begin{table}[tbh!]
% 	\centering
% 	\caption{Calculated defect formation energies $\Delta H^f$ at $E_f = 0$ for native point defects in ZnO under oxygen-poor and oxygen-rich conditions. Also shown in this table are the defect charge state $q$ and the number of constituent atoms, $n_{\ch{O}}$ and $n_{\ch{Zn}}$, added or removed in the defect supercell.} 
% 	\label{tab:E_formation}
% 	\resizebox{0.9\textwidth}{!}{%m{0.2\textwidth}<{\centering}
% 	\begin{tabular}{@{}cccccc}
% 	\toprule
% 	Defect                & $q$  & $n_{\ch{O}}$ & $n_{\ch{Zn}}$ & $\Delta H^f$ (O-poor) (eV) & $\Delta H^f$ (O-rich) (eV) \\ \midrule
% 	\multirow[t]{3}{*}{$V_{\ch{O}}$} & 0  & -1 & 0   & -0.018      & 3.589       \\
% 						  & 1  & -1 & 0   & -2.950      & 0.657       \\
% 						  & 2  & -1 & 0   & -6.466      & -2.859      \\
% 	\multirow[t]{3}{*}{$V_{\ch{Zn}}$}  & 0  & 0  & -1  & 9.821       & 6.214       \\
% 						  & -1 & 0  & -1  & 10.474      & 6.867       \\
% 						  & -2 & 0  & -1  & 12.004      & 8.396       \\
% 	\ch{O}$_i$                    & 0  & 1  & 0   & 11.558      & 7.951       \\
% 	\multirow[t]{3}{*}{\ch{Zn}$_i$}               & 0  & 0  & 1   & 4.277       & 7.884       \\
% 						  & 1  & 0  & 1   & -0.569      & 3.038       \\
% 						  & 2  & 0  & 1   & -4.481      & -0.874      \\
% 	\ch{O}$_{\ch{Zn}}$                    & 0  & 1  & -1  & 15.495      & 8.281       \\
% 	\ch{Zn}$_{\ch{O}}$                   & 0  & -1 & 1   & 3.561       & 10.775      \\ \bottomrule
% 	\end{tabular}%
% 	}
% 	\end{table}
% \endgroup

\begingroup
\setlength{\tabcolsep}{15pt}
\begin{table}[tbhp!]
	\centering
	\caption{Calculated defect formation energies $\Delta H^f$ at $E_f = 0$ and $E_f = E_g$ for native point defects in ZnO under oxygen-poor and oxygen-rich conditions. Also shown in this table are the defect charge state $q$ and the number of constituent atoms, $n_{\ch{O}}$ and $n_{\ch{Zn}}$, added or removed in the defect supercell.}
	\label{tab:E_formation}
	\resizebox{0.9\textwidth}{!}{%
	\begin{tabular}{@{}cccccccccc@{}}
	\toprule
	\multirow{2}{*}{Defect} & \multirow{2}{*}{q} & \multirow{2}{*}{$n_{\ch{O}}$} & \multirow{2}{*}{$n_{\ch{Zn}}$} &  & \multicolumn{2}{c}{$\Delta H^f$ (O-poor) (eV)} &  & \multicolumn{2}{c}{ $\Delta H^f$ (O-rich) (eV)} \\ \cmidrule(lr){6-7} \cmidrule(l){9-10} 
							&                    &                     &                      &  & $E_f = 0$            & $E_f = E_g$           &  & $E_f = 0$           & $E_f = E_g$          \\ \midrule
	$V_{\ch{O}}$                    & 0                  & -1                  & 0                    &  & -0.018         & -0.018         &  & 3.589          & 3.589          \\
							& 1                  & -1                  & 0                    &  & -2.950         & 0.440          &  & 0.657          & 4.047          \\
							& 2                  & -1                  & 0                    &  & -6.466         & 0.314          &  & -2.859         & 3.921          \\
	\multirow{3}{*}{$V_{\ch{Zn}}$}    & 0                  & 0                   & -1                   &  & 9.821          & 9.821          &  & 6.214          & 6.214          \\
							& -1                 & 0                   & -1                   &  & 10.474         & 7.084          &  & 6.867          & 3.477          \\
							& -2                 & 0                   & -1                   &  & 12.004         & 5.224          &  & 8.396          & 1.616          \\
	\ch{O}$_i$                      & 0                  & 1                   & 0                    &  & 11.558         & 11.558         &  & 7.951          & 7.951          \\
	\multirow{3}{*}{\ch{Zn}$_i$}    & 0                  & 0                   & 1                    &  & 4.277          & 4.277          &  & 7.884          & 7.884          \\
							& 1                  & 0                   & 1                    &  & -0.569         & 2.821          &  & 3.038          & 6.428          \\
							& 2                  & 0                   & 1                    &  & -4.481         & 2.299          &  & -0.874         & 5.906          \\
	\ch{O}$_{\ch{Zn}}$                      & 0                  & 1                   & -1                   &  & 15.495         & 15.495         &  & 8.281          & 8.281          \\
	\ch{Zn}$_{\ch{O}}$                    & 0                  & -1                  & 1                    &  & 3.561          & 3.561          &  & 10.775         & 10.775         \\ \bottomrule
	\end{tabular}%
	}
	\end{table}
\endgroup

\begin{figure}[tbh!]
	\centering
	\includegraphics[width=0.4\linewidth]{"images/rnd/trans_lvl"}
	\caption[Thermodynamic transition levels for defects in ZnO]{Thermodynamic transition levels for defects in ZnO. The horizontal dashed line is the band gap.}
	%\label{fig:free-electron}
\end{figure}


\clearpage

\section{Oxygen Vacancies}

\begin{figure}[tbh!]
	\centering
	\includegraphics[width=0.6\linewidth]{"images/rnd/band-dos_O_vac"}
	\caption[Bandstructure of Oxygen vacancy]{Bandstructure of Oxygen vacancy}
	%\label{fig:free-electron}
\end{figure}

\begin{figure}[tbh!]
	\centering
	\includegraphics[width=0.6\linewidth]{"images/rnd/dos-pdos_O_vac"}
	\caption[Partial density of states of Oxygen vacancy]{Partial density of states of Oxygen vacancy}
	%\label{fig:free-electron}
\end{figure}

% +1 Oxygen vacancy
\begin{figure}[tbh!]
	\centering
	\includegraphics[width=0.6\linewidth]{"images/rnd/band-dos_O_vac-p1"}
	\caption[Bandstructure of +1 Oxygen vacancy]{Bandstructure of +1 Oxygen vacancy}
	%\label{fig:free-electron}
\end{figure}

\begin{figure}[tbh!]
	\centering
	\includegraphics[width=0.6\linewidth]{"images/rnd/dos-pdos_O_vac-p1"}
	\caption[Partial density of states of +1 Oxygen vacancy]{Partial density of states of +1 Oxygen vacancy}
	%\label{fig:free-electron}
\end{figure}

% +2 Oxygen vacancy
\begin{figure}[tbh!]
	\centering
	\includegraphics[width=0.6\linewidth]{"images/rnd/band-dos_O_vac-p2"}
	\caption[Bandstructure of +2 Oxygen vacancy]{Bandstructure of +2 Oxygen vacancy}
	%\label{fig:free-electron}
\end{figure}

\begin{figure}[tbh!]
	\centering
	\includegraphics[width=0.6\linewidth]{"images/rnd/dos-pdos_O_vac-p2"}
	\caption[Partial density of states of +2 Oxygen vacancy]{Partial density of states of +2 Oxygen vacancy}
	%\label{fig:free-electron}
\end{figure}

\clearpage


\section{Zinc Vacancies}
\begin{figure}[tbh!]
	\centering
	\includegraphics[width=0.6\linewidth]{"images/rnd/band-dos_Zn_vac"}
	\caption[Bandstructure of Zinc vacancy]{Bandstructure of Zinc vacancy}
	%\label{fig:free-electron}
\end{figure}

\begin{figure}[tbh!]
	\centering
	\includegraphics[width=0.6\linewidth]{"images/rnd/dos-pdos_Zn_vac"}
	\caption[Partial density of states of Zinc vacancy]{Partial density of states of Zinc vacancy}
	%\label{fig:free-electron}
\end{figure}

\begin{figure}[tbh!]
	\centering
	\includegraphics[width=0.6\linewidth]{"images/rnd/band-dos-close_Zn_vac"}
	\caption[Bandstructure of Zinc vacancy near the valence band]{Bandstructure of Zinc vacancy near the valence band}
	%\label{fig:free-electron}
\end{figure}

% -1 vinc vacancy
\begin{figure}[tbh!]
	\centering
	\includegraphics[width=0.6\linewidth]{"images/rnd/band-dos_Zn_vac-n1"}
	\caption[Bandstructure of -1 Zinc vacancy]{Bandstructure of -1 Zinc vacancy}
	%\label{fig:free-electron}
\end{figure}

\begin{figure}[tbh!]
	\centering
	\includegraphics[width=0.6\linewidth]{"images/rnd/dos-pdos_Zn_vac-n1"}
	\caption[Partial density of states of -1 Zinc vacancy]{Partial density of states of -1 Zinc vacancy}
	%\label{fig:free-electron}
\end{figure}

\begin{figure}[tbh!]
	\centering
	\includegraphics[width=0.6\linewidth]{"images/rnd/band-dos-close_Zn_vac-n1"}
	\caption[Bandstructure of -1 Zinc vacancy near the valence band]{Bandstructure of -1 Zinc vacancy near the valence band}
	%\label{fig:free-electron}
\end{figure}

% -2 vinc vacancy
\begin{figure}[tbh!]
	\centering
	\includegraphics[width=0.6\linewidth]{"images/rnd/band-dos_Zn_vac-n2"}
	\caption[Bandstructure of -2 Zinc vacancy]{Bandstructure of -2 Zinc vacancy}
	%\label{fig:free-electron}
\end{figure}

\begin{figure}[tbh!]
	\centering
	\includegraphics[width=0.6\linewidth]{"images/rnd/dos-pdos_Zn_vac-n2"}
	\caption[Partial density of states of -2 Zinc vacancy]{Partial density of states of -2 Zinc vacancy}
	%\label{fig:free-electron}
\end{figure}

\begin{figure}[tbh!]
	\centering
	\includegraphics[width=0.6\linewidth]{"images/rnd/band-dos-close_Zn_vac-n2"}
	\caption[Bandstructure of -2 Zinc vacancy near the valence band]{Bandstructure of -2 Zinc vacancy near the valence band}
	%\label{fig:free-electron}
\end{figure}

\clearpage

\section{Oxygen Interstitials}
will add charge defects

\begin{figure}[tbh!]
	\centering
	\includegraphics[width=0.6\linewidth]{"images/rnd/band-dos_O_i"}
	\caption[Bandstructure of Oxygen interstitial]{Bandstructure of Oxygen interstitial}
	%\label{fig:free-electron}
\end{figure}

\begin{figure}[tbh!]
	\centering
	\includegraphics[width=0.6\linewidth]{"images/rnd/dos-pdos_O_i"}
	\caption[Partial density of states of Oxygen interstitial]{Partial density of states of Oxygen interstitial}
	%\label{fig:free-electron}
\end{figure}

\begin{figure}[tbh!]
	\centering
	\includegraphics[width=0.6\linewidth]{"images/rnd/band-dos-close_O_i"}
	\caption[Bandstructure of Oxygen interstitial near the valence band]{Bandstructure of Oxygen interstitial near the valence band}
	%\label{fig:free-electron}
\end{figure}

\clearpage

\section{Zinc Interstitials}
% Zinc 

For Zn interstitial, the octahedral site was considered in this study. The tetrahedral site is reported to be energetically less favorable or unstable. 

\begin{figure}[tbh!]
	\centering
	\includegraphics[width=0.6\linewidth]{"images/rnd/band-dos_Zn_i"}
	\caption[Bandstructure of Zinc interstitial]{Bandstructure of Zinc interstitial}
	%\label{fig:free-electron}
\end{figure}

\begin{figure}[tbh!]
	\centering
	\includegraphics[width=0.6\linewidth]{"images/rnd/dos-pdos_Zn_i"}
	\caption[Partial density of states of Zinc interstitial]{Partial density of states of Zinc interstitial}
	%\label{fig:free-electron}
\end{figure}

% \begin{figure}[tbh!]
% 	\centering
% 	\includegraphics[width=0.5\linewidth]{"images/rnd/dos-pdos-zoom_Zn_i"}
% 	\caption[Partial density of states of Zinc interstitial near Conduction Band]{Partial density of states of Zinc interstitial near Conduction Band}
% 	%\label{fig:free-electron}
% \end{figure}

% +1 Zinc interstitial 
\begin{figure}[tbh!]
	\centering
	\includegraphics[width=0.6\linewidth]{"images/rnd/band-dos_Zn_i-p1"}
	\caption[Bandstructure of +1 Zinc interstitial]{Bandstructure of +1 Zinc interstitial}
	%\label{fig:free-electron}
\end{figure}

\begin{figure}[tbh!]
	\centering
	\includegraphics[width=0.6\linewidth]{"images/rnd/dos-pdos_Zn_i-p1"}
	\caption[Partial density of states of +1 Zinc interstitial]{Partial density of states of +1 Zinc interstitial }
	%\label{fig:free-electron}
\end{figure}

% +2 Zinc interstitial 

\begin{figure}[tbh!]
	\centering
	\includegraphics[width=0.6\linewidth]{"images/rnd/band-dos_Zn_i-p2"}
	\caption[Bandstructure of +2 Zinc interstitial]{Bandstructure of +2 Zinc interstitial}
	%\label{fig:free-electron}
\end{figure}

\begin{figure}[tbh!]
	\centering
	\includegraphics[width=0.6\linewidth]{"images/rnd/dos-pdos_Zn_i-p2"}
	\caption[Partial density of states of +2 Zinc interstitial]{Partial density of states of +2 Zinc interstitial }
	%\label{fig:free-electron}
\end{figure}


\clearpage

\section{Oxygen Antisites}

\begin{figure}[tbh!]
	\centering
	\includegraphics[width=0.6\linewidth]{"images/rnd/band-dos_O_anti"}
	\caption[Bandstructure of Oxygen antisite]{Bandstructure of Oxygen antisite}
	%\label{fig:free-electron}
\end{figure}

\begin{figure}[tbh!]
	\centering
	\includegraphics[width=0.6\linewidth]{"images/rnd/dos-pdos_O_anti"}
	\caption[Partial density of states of Oxygen antisite]{Partial density of states of Oxygen antisite}
	%\label{fig:free-electron}
\end{figure}

\begin{figure}[tbh!]
	\centering
	\includegraphics[width=0.6\linewidth]{"images/rnd/band-dos-close_O_anti"}
	\caption[Bandstructure of Oxygen antisite in the band gap region]{Bandstructure of Oxygen antisite  in the band gap region}
	%\label{fig:free-electron}
\end{figure}

\section{Zinc Antisites}
Zinc antisite can be thought as combination of oxygen vacancy and zinc interstitial. This is reasonable since an oxygen atom must be removed first before adding the zinc atom on the same location.  Thus in the bandstructure diagram shown in Figure \ref{fig:bands.Zn-anti}, two defects can be observed. One is in the lower part of the band gap that resembles the defect state of oxygen vacancy, and the other is a state resonant with the conduction band that resembles the zinc interstitial state \citep{Janotti2007}. The existence of the resonant state can be shown by a DOS plot in Figure \ref{fig:dos.Zn-anti} where a notable peak is observed in the bottom part of the conduction band. The electrons in the resonant state can occupy the CBM that causes the zinc antisite to act as a shallow donor. 

\begin{figure}[tbh!]
	\centering
	\includegraphics[width=0.6\linewidth]{"images/rnd/band-dos_Zn_anti"}
	\caption[Bandstructure of Zinc antisite]{Bandstructure of Zinc antisite}
	\label{fig:bands.Zn-anti}
\end{figure}

\begin{figure}[tbh!]
	\centering
	\includegraphics[width=0.6\linewidth]{"images/rnd/dos-pdos_Zn_anti"}
	\caption[Partial density of states of Zinc antisite]{Partial density of states of Zinc antisite}
	\label{fig:dos.Zn-anti}
\end{figure}


