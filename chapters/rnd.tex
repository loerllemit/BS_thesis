\chapter{Results and Discussion} \label{chap:rnd}
\section{Convergence Tests}
Convergence tests on the ZnO unit cell were done by varying the k-points, and the kinetic energy cut-offs  of the Kohn-Sham orbital and the electron density, respectively. When the k-points are varied, the cut-off energies  were fixed by the minimum energy set by the pseudopotential used. In this case  As shown in Figure in \ref{fig:conv_k}, the pressure and total energy converges at $k \geq 2$ which corresponds to the Monkhorst-Pack grid of $7 \times 7 \times 2$. One can choose the minimum k-point required for convergence to be used for subsequent calculations. However, higher values are preferred as to make sure the convergence is guaranteed for all configuration of the system. In this thesis, the k-point used was $k = 6$ which corresponds to the Monkhorst-Pack grid $11 \times 11 \times 6$. Next, 

\begin{figure}[tbh!] 
	\centering
	\includegraphics[width=0.7\linewidth]{"images/rnd/kpoints"}
	\caption[Convergence of pressure and total energies with respect to k-points]{Convergence of pressure and total energies with respect to k-points. The x-axis denotes the k-point $k$ in the Monkhorst-Pack grid of $(k+5) \times (k +5 ) \times k$}
	\label{fig:conv_k}
\end{figure}

\begin{figure}[tbh!]
	\centering
	\includegraphics[width=0.7\linewidth]{"images/rnd/ecutwfc"}
	\caption[Convergence of pressure and total energies with respect to cutoff energy of the Kohn-Sham orbital]{Convergence of pressure and total energies with respect to cutoff energy of the Kohn-Sham orbital}
\end{figure}

\begin{figure}[tbh!]
	\centering
	\includegraphics[width=0.7\linewidth]{"images/rnd/ecutrho"}
	\caption[Convergence of pressure and total energies with respect to cutoff energy of the electron density]{Convergence of pressure and total energies with respect to cutoff energy of the electron density}
\end{figure}

\section{Hubbard-U parameters on band gap variation}
The influence of both $U_p$, acting on O 2p orbitals,  and $U_d$, acting on Zn 3d orbitals, on the p-d hybridization is evident in the variation of band gap as shown in Figure \ref{fig:hubbard_val}. Huge increase in the magnitude of the electronic band gap of almost 0.2 eV with the increment of 0.5 eV in the value of $U_p$. 


Since standard DFT such as GGA  underestimates the band gaps, the position of the defect states is also underestimated.

\begin{figure}[tbh!]
	\centering
	\includegraphics[width=0.7\linewidth]{"images/rnd/hubbard_val"}
	\caption[Variation of band gap with respect to Hubbard parameter $U_p$ applied to O 2p orbitals for a given $U_d$ applied to Zn 3d orbitals. ]{Variation of band gap with respect to Hubbard parameter $U_p$ applied to O 2p orbitals for a given $U_d$ values applied to Zn 3d orbitals. The horizontal dashed line indicates the experimental band gap of 3.37 eV.}
	\label{fig:hubbard_val}
\end{figure}


\section{Geometric Relaxation}
 The calculated equilibrium lattice parameters and the band gaps using DFT and DFT+U are listed in Table blah.  
 
change in lattice parameters

For Zn interstitial, the octahedral site was considered in this study. The tetrahedral site is reported to be energetically less favorable or unstable. 
\clearpage

\section{Formation Energies}

\section{Vacancies}

\begin{figure}[tbh!]
	\centering
	\includegraphics[width=0.7\linewidth]{"images/rnd/band-dos_O_vac"}
	\caption[Bandstructure of Oxygen vacancy]{Bandstructure of Oxygen vacancy}
	%\label{fig:free-electron}
\end{figure}

\begin{figure}[tbh!]
	\centering
	\includegraphics[width=0.7\linewidth]{"images/rnd/dos-pdos_O_vac"}
	\caption[Partial density of states of Oxygen vacancy]{Partial density of states of Oxygen vacancy}
	%\label{fig:free-electron}
\end{figure}


\begin{figure}[tbh!]
	\centering
	\includegraphics[width=0.7\linewidth]{"images/rnd/band-dos_Zn_vac"}
	\caption[Bandstructure of Zinc vacancy]{Bandstructure of Zinc vacancy}
	%\label{fig:free-electron}
\end{figure}

\begin{figure}[tbh!]
	\centering
	\includegraphics[width=0.7\linewidth]{"images/rnd/dos-pdos_Zn_vac"}
	\caption[Partial density of states of Zinc vacancy]{Partial density of states of Zinc vacancy}
	%\label{fig:free-electron}
\end{figure}

\begin{figure}[tbh!]
	\centering
	\includegraphics[width=0.7\linewidth]{"images/rnd/band-dos-close_Zn_vac"}
	\caption[Bandstructure of Zinc vacancy near the valence band]{Bandstructure of Zinc vacancy near the valence band}
	%\label{fig:free-electron}
\end{figure}

\clearpage

\section{Interstitials}

\begin{figure}[tbh!]
	\centering
	\includegraphics[width=0.7\linewidth]{"images/rnd/band-dos_O_i"}
	\caption[Bandstructure of Oxygen interstitial]{Bandstructure of Oxygen interstitial}
	%\label{fig:free-electron}
\end{figure}

\begin{figure}[tbh!]
	\centering
	\includegraphics[width=0.7\linewidth]{"images/rnd/dos-pdos_O_i"}
	\caption[Partial density of states of Oxygen interstitial]{Partial density of states of Oxygen interstitial}
	%\label{fig:free-electron}
\end{figure}

\begin{figure}[tbh!]
	\centering
	\includegraphics[width=0.7\linewidth]{"images/rnd/band-dos-close_O_i"}
	\caption[Bandstructure of Oxygen interstitial near the valence band]{Bandstructure of Oxygen interstitial near the valence band}
	%\label{fig:free-electron}
\end{figure}


% Zinc 
\begin{figure}[tbh!]
	\centering
	\includegraphics[width=0.7\linewidth]{"images/rnd/band-dos_Zn_i"}
	\caption[Bandstructure of Zinc interstitial]{Bandstructure of Zinc interstitial}
	%\label{fig:free-electron}
\end{figure}

\begin{figure}[tbh!]
	\centering
	\includegraphics[width=0.7\linewidth]{"images/rnd/dos-pdos_Zn_i"}
	\caption[Partial density of states of Zinc interstitial]{Partial density of states of Zinc interstitial}
	%\label{fig:free-electron}
\end{figure}

\begin{figure}[tbh!]
	\centering
	\includegraphics[width=0.5\linewidth]{"images/rnd/dos-pdos-zoom_Zn_i"}
	\caption[Partial density of states of Zinc interstitial near Conduction Band]{Partial density of states of Zinc interstitial near Conduction Band}
	%\label{fig:free-electron}
\end{figure}

\clearpage
\section{Antisites}

\begin{figure}[tbh!]
	\centering
	\includegraphics[width=0.7\linewidth]{"images/rnd/band-dos_O_anti"}
	\caption[Bandstructure of Oxygen antisite]{Bandstructure of Oxygen antisite}
	%\label{fig:free-electron}
\end{figure}

\begin{figure}[tbh!]
	\centering
	\includegraphics[width=0.7\linewidth]{"images/rnd/dos-pdos_O_anti"}
	\caption[Partial density of states of Oxygen antisite]{Partial density of states of Oxygen antisite}
	%\label{fig:free-electron}
\end{figure}

\begin{figure}[tbh!]
	\centering
	\includegraphics[width=0.7\linewidth]{"images/rnd/band-dos-close_O_anti"}
	\caption[Bandstructure of Oxygen antisite in the band gap region]{Bandstructure of Oxygen antisite  in the band gap region}
	%\label{fig:free-electron}
\end{figure}

Zinc antisite can be thought as combination of oxygen vacancy and zinc interstitial. This is reasonable since an oxygen atom must be removed first before adding the zinc atom on the same location.  Thus in the bandstructure diagram shown in Figure \ref{fig:bands.Zn-anti}, two defects can be observed. One is in the lower part of the band gap that resembles the defect state of oxygen vacancy, and the other is a state resonant with the conduction band that resembles the zinc interstitial state \citep{Janotti2007}. The existence of the resonant state can be shown by a DOS plot in Figure \ref{fig:dos.Zn-anti} where a notable peak is observed in the bottom part of the conduction band. The electrons in the resonant state can occupy the CBM that causes the zinc antisite to act as a shallow donor. 

\begin{figure}[tbh!]
	\centering
	\includegraphics[width=0.7\linewidth]{"images/rnd/band-dos_Zn_anti"}
	\caption[Bandstructure of Zinc antisite]{Bandstructure of Zinc antisite}
	\label{fig:bands.Zn-anti}
\end{figure}

\begin{figure}[tbh!]
	\centering
	\includegraphics[width=0.7\linewidth]{"images/rnd/dos-pdos_Zn_anti"}
	\caption[Partial density of states of Zinc antisite]{Partial density of states of Zinc antisite}
	\label{fig:dos.Zn-anti}
\end{figure}


\clearpage
\section{Charged Defects}
% +1 Oxygen vacancy
\begin{figure}[tbh!]
	\centering
	\includegraphics[width=0.7\linewidth]{"images/rnd/band-dos_O_vac-p1"}
	\caption[Bandstructure of +1 Oxygen vacancy]{Bandstructure of +1 Oxygen vacancy}
	%\label{fig:free-electron}
\end{figure}

\begin{figure}[tbh!]
	\centering
	\includegraphics[width=0.7\linewidth]{"images/rnd/dos-pdos_O_vac-p1"}
	\caption[Partial density of states of +1 Oxygen vacancy]{Partial density of states of +1 Oxygen vacancy}
	%\label{fig:free-electron}
\end{figure}

% +2 Oxygen vacancy
\begin{figure}[tbh!]
	\centering
	\includegraphics[width=0.7\linewidth]{"images/rnd/band-dos_O_vac-p2"}
	\caption[Bandstructure of +2 Oxygen vacancy]{Bandstructure of +2 Oxygen vacancy}
	%\label{fig:free-electron}
\end{figure}

\begin{figure}[tbh!]
	\centering
	\includegraphics[width=0.7\linewidth]{"images/rnd/dos-pdos_O_vac-p2"}
	\caption[Partial density of states of +2 Oxygen vacancy]{Partial density of states of +2 Oxygen vacancy}
	%\label{fig:free-electron}
\end{figure}

% +1 Zinc interstitial 
\begin{figure}[tbh!]
	\centering
	\includegraphics[width=0.7\linewidth]{"images/rnd/band-dos_Zn_i-p1"}
	\caption[Bandstructure of +1 Zinc interstitial]{Bandstructure of +1 Zinc interstitial}
	%\label{fig:free-electron}
\end{figure}

\begin{figure}[tbh!]
	\centering
	\includegraphics[width=0.7\linewidth]{"images/rnd/dos-pdos_Zn_i-p1"}
	\caption[Partial density of states of +1 Zinc interstitial]{Partial density of states of +1 Zinc interstitial }
	%\label{fig:free-electron}
\end{figure}

\begin{figure}[tbh!]
	\centering
	\includegraphics[width=0.5\linewidth]{"images/rnd/dos-pdos-zoom_Zn_i-p1"}
	\caption[Partial density of states of +1 Zinc interstitial near Conduction Band]{Partial density of states of +1 Zinc interstitial near Conduction Band}
	%\label{fig:free-electron}
\end{figure}

% +2 Zinc interstitial 

\begin{figure}[tbh!]
	\centering
	\includegraphics[width=0.7\linewidth]{"images/rnd/band-dos_Zn_i-p2"}
	\caption[Bandstructure of +2 Zinc interstitial]{Bandstructure of +2 Zinc interstitial}
	%\label{fig:free-electron}
\end{figure}

\begin{figure}[tbh!]
	\centering
	\includegraphics[width=0.7\linewidth]{"images/rnd/dos-pdos_Zn_i-p2"}
	\caption[Partial density of states of +2 Zinc interstitial]{Partial density of states of +2 Zinc interstitial }
	%\label{fig:free-electron}
\end{figure}

\begin{figure}[tbh!]
	\centering
	\includegraphics[width=0.5\linewidth]{"images/rnd/dos-pdos-zoom_Zn_i-p2"}
	\caption[Partial density of states of +2 Zinc interstitial near Conduction Band]{Partial density of states of +2 Zinc interstitial near Conduction Band}
	%\label{fig:free-electron}
\end{figure}


