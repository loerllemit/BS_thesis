\chapter{Results and Discussion} \label{chap:rnd}
\section{Convergence Tests}
Convergence tests on the ZnO unit cell were done by varying the k-points, and the kinetic energy cut-offs  of the Kohn-Sham orbital and the electron density, respectively. When the k-points are varied, the cut-off energies  were fixed by the minimum energy set by the pseudopotential used. In this case  As shown in Figure in \ref{fig:conv_k}, the pressure and total energy converges at $k \geq 2$ which corresponds to the Monkhorst-Pack grid of $7 \times 7 \times 2$. One can choose the minimum k-point required for convergence to be used for subsequent calculations. However, higher values are preferred as to make sure the convergence is guaranteed for all configuration of the system. In this thesis, the k-point used was $k = 6$ which corresponds to the Monkhorst-Pack grid $11 \times 11 \times 6$. Next, 

\begin{figure}[tbh!] 
	\centering
	\includegraphics[width=0.7\linewidth]{"images/rnd/kpoints"}
	\caption[Convergence of pressure and total energies with respect to k-points]{Convergence of pressure and total energies with respect to k-points. The x-axis denotes the k-point $k$ in the Monkhorst-Pack grid of $(k+5) \times (k +5 ) \times k$}
	\label{fig:conv_k}
\end{figure}

\begin{figure}[tbh!]
	\centering
	\includegraphics[width=0.7\linewidth]{"images/rnd/ecutwfc"}
	\caption[Convergence of pressure and total energies with respect to cutoff energy of the Kohn-Sham orbital]{Convergence of pressure and total energies with respect to cutoff energy of the Kohn-Sham orbital}
\end{figure}

\begin{figure}[tbh!]
	\centering
	\includegraphics[width=0.7\linewidth]{"images/rnd/ecutrho"}
	\caption[Convergence of pressure and total energies with respect to cutoff energy of the electron density]{Convergence of pressure and total energies with respect to cutoff energy of the electron density}
\end{figure}

\clearpage
\section{Hubbard-U parameters}
The influence of both $U_p$, acting on O 2p orbitals,  and $U_d$, acting on Zn 3d orbitals, on the p-d hybridization is evident in the variation of band gap as shown in Figure \ref{fig:hubbard_val}. For a fixed $U_d$, there is a huge increase in the magnitude of the electronic band gap of almost 0.2 eV with the increment of $U_p$ by 0.5 eV. The rate of increase in band gap is high enough that most band gap values have exceeded the reference band gap, shown as a horizontal dashed line in Figure \ref{fig:hubbard_val}.  On the other hand,  the band gap increases steadily with $U_d$ for a given $U_p$. The higher the $U_d$ value, the lower $U_p$ needed to widen the band gap. In either case, the band gap monotonically increase with both parameters. The observed trend agrees very well with the recent results of Goh et al. \citep{Goh2017}. 


Among the values, $U_d = 15$ eV and $U_p = 7$ eV has the band gap  value of 3.395 eV which is closest to the experimental value of 3.37 eV. Other combinations of $U_d$ and $U_p$ listed in Table \ref{tab:hubbard_table} are equally valid in correcting the band gap. Nevertheless, this study chose $U_d = 15$ eV and $U_p = 7$ eV as the correction parameters since the calculated band gap falls between the experimental band gap of 3.37 eV to 3.44 eV. Standard DFT calculations using GGA will produce severe underestimation of the band gap with  a value of 0.7589 eV, an error of more than 75\%, consistent with the reports of Haq et al. \citep{Haq2013} and Erhart et al. \citep{Erhart2006a}. This underestimation will affect energetics of the defect states and will lead to erroneous results.

For the structural relaxation, the calculated equilibrium lattice parameters and the unit cell volume  using DFT-GGA and DFT+U are listed in Table \ref{tab:hubbard_table}.  Both lattice parameter $a$ and $c$ increase with $U_d$ values regardless of $U_p$. However, the change is in order of 0.001 \AA \ which is relatively small and can be tolerated. Hence, DFT-U calculations do not substantially alter the lattice parameters. The c/a ratio determines the elongation of wurtzite ZnO along the c-axis. A decreasing trend of c/a ratio can be observed with increasing $U_d$. Ma et al. \citep{Ma2013} suggested that a strong on-site correction causes the localization of Zn 3d-states which further attracts the electron toward the core resulting to  compression of wurtzite ZnO. The calculated unit cell volume does not deviate by more than 1.5\%. The lattice parameters calculated using GGA are overestimate by 1.28\% and 2\% for lattice parameters $a$ and $c$, respectively. The calculated c/a ratio and unit cell volume are higher compared to the experimental values. However, the discrepancies in GGA calculations are considered to be tolerable and are still in good agreement with experimental data. Since the associated computational expense in structural relaxation, DFT+U relaxations will not be used. GGA will be used instead in structural relaxation of defect structures of ZnO  due to its computational simplicity and errors are generally tolerable.

An energy bandstructure diagram in Figure \ref{fig:hubbard_band-dos} shows the shifting of the conduction band upwards with the application of Hubbard-U correction. The valence band consists of two separated regions, with widths of blah blah, respectively. Standard GGA calculations resulted to the overlapping of these two regions  which are separated in DFT+U calculation. The conduction band minimum (CBM) is exactly similar in terms of width and dispersion for both calculations which differed only by a translation of energy due to widening of band gap. Also, the top of the valence band shows similar dispersion in both calculations. The observed similarities indicate that the effective masses of electrons and holes  are not affected by the DFT+U correction \citep{Goh2017}. 





\begin{figure}[htbp!]
	\centering
	\includegraphics[width=0.7\linewidth]{"images/rnd/hubbard_val"}
	\caption[Variation of band gap with respect to Hubbard parameter $U_p$ applied to O 2p orbitals for a given $U_d$ applied to Zn 3d orbitals. ]{Variation of band gap with respect to Hubbard parameter $U_p$ applied to O 2p orbitals for a given $U_d$ values applied to Zn 3d orbitals. The horizontal dashed line indicates the experimental band gap of 3.37 eV.}
	\label{fig:hubbard_val}
\end{figure}

\begin{table}[htbp!]
	\centering
	\caption{Comparison of  structural parameters and electronic band gap using different DFT methods. }
	\resizebox{0.9\textwidth}{!}{%
	\begin{threeparttable}[t]
	\begin{tabular}{@{}cccccccc@{}}
	\toprule
	Method & $U_d$ (eV) & $U_p$ (eV)  & a (\AA)     & c (\AA)     & c/a    & Volume (\AA$^3$) & band gap (eV) \\ \midrule
	DFT+U  & 11 & 7.5 & 3.2458 & 5.2137 & 1.6063 & 47.5685 & 3.4085   \\
	DFT+U  & 13 & 7.0 & 3.2501 & 5.2197 & 1.6060 & 47.7503 & 3.3344   \\
	DFT+U  & 15 & 7.0 & 3.2517 & 5.2211 & 1.6057 & 47.8092 & 3.3950    \\
	DFT+U  & 17 & 6.5 & 3.2557 & 5.2272 & 1.6056 & 47.9827 & 3.2897   \\
	DFT+U  & 19 & 6.5 & 3.2569 & 5.2284 & 1.6053 & 48.0309 & 3.3312   \\
	GGA    &  -  &  -   & 3.2842 & 5.2982 & 1.6132 & 49.4886 & 0.7589   \\
	expt.\tnote{a} &  -  &   -  & 3.2427 & 5.1948 & 1.6020 & 47.3057 & 3.3700     \\ \bottomrule
	\end{tabular}
	\vspace{-8pt}
	\begin{tablenotes}[para]
		\footnotesize
		\item[a] structural parameters were obtained from \citep{Sabine1969}, whereas the band gap value is from  \citep{Chitra2020}
	\end{tablenotes}
	\end{threeparttable}
	}
	\label{tab:hubbard_table}
	\end{table}

	\begin{figure}[tbh!]
		\centering
		\includegraphics[width=0.8\linewidth]{"images/rnd/band-dos_juxtapose"}
		\caption[Energy bandstructures and total density of states obtained from standard DFT-GGA calculation (blue dashed lines) and DFT+U calculation (red solid lines)with $U_d =15$ and $U_p= 7$]{Energy bandstructures and total density of states obtained from standard DFT-GGA calculation (blue dashed lines) and DFT+U calculation (red solid lines) with $U_d =15$ and $U_p= 7$. Fermi level is shifted to zero indicated as horizontal dotted line. }
		\label{fig:hubbard_band-dos}
	\end{figure}


\clearpage

\section{Formation Energies}

\section{Vacancies}

\begin{figure}[tbh!]
	\centering
	\includegraphics[width=0.7\linewidth]{"images/rnd/band-dos_O_vac"}
	\caption[Bandstructure of Oxygen vacancy]{Bandstructure of Oxygen vacancy}
	%\label{fig:free-electron}
\end{figure}

\begin{figure}[tbh!]
	\centering
	\includegraphics[width=0.7\linewidth]{"images/rnd/dos-pdos_O_vac"}
	\caption[Partial density of states of Oxygen vacancy]{Partial density of states of Oxygen vacancy}
	%\label{fig:free-electron}
\end{figure}


\begin{figure}[tbh!]
	\centering
	\includegraphics[width=0.7\linewidth]{"images/rnd/band-dos_Zn_vac"}
	\caption[Bandstructure of Zinc vacancy]{Bandstructure of Zinc vacancy}
	%\label{fig:free-electron}
\end{figure}

\begin{figure}[tbh!]
	\centering
	\includegraphics[width=0.7\linewidth]{"images/rnd/dos-pdos_Zn_vac"}
	\caption[Partial density of states of Zinc vacancy]{Partial density of states of Zinc vacancy}
	%\label{fig:free-electron}
\end{figure}

\begin{figure}[tbh!]
	\centering
	\includegraphics[width=0.7\linewidth]{"images/rnd/band-dos-close_Zn_vac"}
	\caption[Bandstructure of Zinc vacancy near the valence band]{Bandstructure of Zinc vacancy near the valence band}
	%\label{fig:free-electron}
\end{figure}

\clearpage

\section{Interstitials}

\begin{figure}[tbh!]
	\centering
	\includegraphics[width=0.7\linewidth]{"images/rnd/band-dos_O_i"}
	\caption[Bandstructure of Oxygen interstitial]{Bandstructure of Oxygen interstitial}
	%\label{fig:free-electron}
\end{figure}

\begin{figure}[tbh!]
	\centering
	\includegraphics[width=0.7\linewidth]{"images/rnd/dos-pdos_O_i"}
	\caption[Partial density of states of Oxygen interstitial]{Partial density of states of Oxygen interstitial}
	%\label{fig:free-electron}
\end{figure}

\begin{figure}[tbh!]
	\centering
	\includegraphics[width=0.7\linewidth]{"images/rnd/band-dos-close_O_i"}
	\caption[Bandstructure of Oxygen interstitial near the valence band]{Bandstructure of Oxygen interstitial near the valence band}
	%\label{fig:free-electron}
\end{figure}

% Zinc 

For Zn interstitial, the octahedral site was considered in this study. The tetrahedral site is reported to be energetically less favorable or unstable. 

\begin{figure}[tbh!]
	\centering
	\includegraphics[width=0.7\linewidth]{"images/rnd/band-dos_Zn_i"}
	\caption[Bandstructure of Zinc interstitial]{Bandstructure of Zinc interstitial}
	%\label{fig:free-electron}
\end{figure}

\begin{figure}[tbh!]
	\centering
	\includegraphics[width=0.7\linewidth]{"images/rnd/dos-pdos_Zn_i"}
	\caption[Partial density of states of Zinc interstitial]{Partial density of states of Zinc interstitial}
	%\label{fig:free-electron}
\end{figure}

\begin{figure}[tbh!]
	\centering
	\includegraphics[width=0.5\linewidth]{"images/rnd/dos-pdos-zoom_Zn_i"}
	\caption[Partial density of states of Zinc interstitial near Conduction Band]{Partial density of states of Zinc interstitial near Conduction Band}
	%\label{fig:free-electron}
\end{figure}

\clearpage
\section{Antisites}

\begin{figure}[tbh!]
	\centering
	\includegraphics[width=0.7\linewidth]{"images/rnd/band-dos_O_anti"}
	\caption[Bandstructure of Oxygen antisite]{Bandstructure of Oxygen antisite}
	%\label{fig:free-electron}
\end{figure}

\begin{figure}[tbh!]
	\centering
	\includegraphics[width=0.7\linewidth]{"images/rnd/dos-pdos_O_anti"}
	\caption[Partial density of states of Oxygen antisite]{Partial density of states of Oxygen antisite}
	%\label{fig:free-electron}
\end{figure}

\begin{figure}[tbh!]
	\centering
	\includegraphics[width=0.7\linewidth]{"images/rnd/band-dos-close_O_anti"}
	\caption[Bandstructure of Oxygen antisite in the band gap region]{Bandstructure of Oxygen antisite  in the band gap region}
	%\label{fig:free-electron}
\end{figure}

Zinc antisite can be thought as combination of oxygen vacancy and zinc interstitial. This is reasonable since an oxygen atom must be removed first before adding the zinc atom on the same location.  Thus in the bandstructure diagram shown in Figure \ref{fig:bands.Zn-anti}, two defects can be observed. One is in the lower part of the band gap that resembles the defect state of oxygen vacancy, and the other is a state resonant with the conduction band that resembles the zinc interstitial state \citep{Janotti2007}. The existence of the resonant state can be shown by a DOS plot in Figure \ref{fig:dos.Zn-anti} where a notable peak is observed in the bottom part of the conduction band. The electrons in the resonant state can occupy the CBM that causes the zinc antisite to act as a shallow donor. 

\begin{figure}[tbh!]
	\centering
	\includegraphics[width=0.7\linewidth]{"images/rnd/band-dos_Zn_anti"}
	\caption[Bandstructure of Zinc antisite]{Bandstructure of Zinc antisite}
	\label{fig:bands.Zn-anti}
\end{figure}

\begin{figure}[tbh!]
	\centering
	\includegraphics[width=0.7\linewidth]{"images/rnd/dos-pdos_Zn_anti"}
	\caption[Partial density of states of Zinc antisite]{Partial density of states of Zinc antisite}
	\label{fig:dos.Zn-anti}
\end{figure}


\clearpage
\section{Charged Defects}
% +1 Oxygen vacancy
\begin{figure}[tbh!]
	\centering
	\includegraphics[width=0.7\linewidth]{"images/rnd/band-dos_O_vac-p1"}
	\caption[Bandstructure of +1 Oxygen vacancy]{Bandstructure of +1 Oxygen vacancy}
	%\label{fig:free-electron}
\end{figure}

\begin{figure}[tbh!]
	\centering
	\includegraphics[width=0.7\linewidth]{"images/rnd/dos-pdos_O_vac-p1"}
	\caption[Partial density of states of +1 Oxygen vacancy]{Partial density of states of +1 Oxygen vacancy}
	%\label{fig:free-electron}
\end{figure}

% +2 Oxygen vacancy
\begin{figure}[tbh!]
	\centering
	\includegraphics[width=0.7\linewidth]{"images/rnd/band-dos_O_vac-p2"}
	\caption[Bandstructure of +2 Oxygen vacancy]{Bandstructure of +2 Oxygen vacancy}
	%\label{fig:free-electron}
\end{figure}

\begin{figure}[tbh!]
	\centering
	\includegraphics[width=0.7\linewidth]{"images/rnd/dos-pdos_O_vac-p2"}
	\caption[Partial density of states of +2 Oxygen vacancy]{Partial density of states of +2 Oxygen vacancy}
	%\label{fig:free-electron}
\end{figure}

% +1 Zinc interstitial 
\begin{figure}[tbh!]
	\centering
	\includegraphics[width=0.7\linewidth]{"images/rnd/band-dos_Zn_i-p1"}
	\caption[Bandstructure of +1 Zinc interstitial]{Bandstructure of +1 Zinc interstitial}
	%\label{fig:free-electron}
\end{figure}

\begin{figure}[tbh!]
	\centering
	\includegraphics[width=0.7\linewidth]{"images/rnd/dos-pdos_Zn_i-p1"}
	\caption[Partial density of states of +1 Zinc interstitial]{Partial density of states of +1 Zinc interstitial }
	%\label{fig:free-electron}
\end{figure}

\begin{figure}[tbh!]
	\centering
	\includegraphics[width=0.5\linewidth]{"images/rnd/dos-pdos-zoom_Zn_i-p1"}
	\caption[Partial density of states of +1 Zinc interstitial near Conduction Band]{Partial density of states of +1 Zinc interstitial near Conduction Band}
	%\label{fig:free-electron}
\end{figure}

% +2 Zinc interstitial 

\begin{figure}[tbh!]
	\centering
	\includegraphics[width=0.7\linewidth]{"images/rnd/band-dos_Zn_i-p2"}
	\caption[Bandstructure of +2 Zinc interstitial]{Bandstructure of +2 Zinc interstitial}
	%\label{fig:free-electron}
\end{figure}

\begin{figure}[tbh!]
	\centering
	\includegraphics[width=0.7\linewidth]{"images/rnd/dos-pdos_Zn_i-p2"}
	\caption[Partial density of states of +2 Zinc interstitial]{Partial density of states of +2 Zinc interstitial }
	%\label{fig:free-electron}
\end{figure}

\begin{figure}[tbh!]
	\centering
	\includegraphics[width=0.5\linewidth]{"images/rnd/dos-pdos-zoom_Zn_i-p2"}
	\caption[Partial density of states of +2 Zinc interstitial near Conduction Band]{Partial density of states of +2 Zinc interstitial near Conduction Band}
	%\label{fig:free-electron}
\end{figure}


