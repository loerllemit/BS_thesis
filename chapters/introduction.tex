\addchapheadtotoc


\chapter{Introduction}
Density Functional Theory proved to be useful and reasonably accurate in determining the physical nature of defects in semiconductors, including  but not limited to native defects, impurities, dopants, and surface defects. 

\section{Purpose and Motivation}
Describe the importance of defects in ZnO
\section{Objectives}
\begin{itemize}
    \item to obtain the bandstructure and density of states of native point defects in wurtzite \ch{ZnO}
    \item to determine which atomic orbitals  contribute to the defect energy level
\end{itemize}

This study considers only the native point defects in ZnO: oxygen and zinc vacancies (\ch{V_O} and \ch{V_{Zn}}), interstitials (\ch{O_i} and \ch{Zn_i}), antisites (\ch{O_{Zn}} and \ch{Zn_O}) and charged defects (\ch{V_O^{1+}}, \ch{V_O^{2+}}, \ch{Zn_i^{1+}} and \ch{Zn_i^{2+}}). 
% Study the mechanisms of different defects in ZnO
\section{Outline}

% \chapter{Acronym Used}
