\addchapheadtotoc


\chapter{Introduction}
Zinc Oxide (ZnO) has remained the focus of rigorous theoretical and experimental studies owing to its unique electronic and optical properties, and its application as an energy sources. ZnO is a wide direct band gap with an experimental value ranging from 3.437 eV at 4 K to  3.37 eV at 300 K emitting at ultraviolet region. In addition, this ionic semiconductor is known to host excitons with a large binding energy of 60 meV, rendering its utility as a UV laser source, tunable UV photodetectors, and scintillator. 

However, realization of a perfect crystal is very difficult or impossible to achieve in experimental conditions due to presence of impurities even at ultra-high vacuum and possibility of forming native defects due to imperfections in the crystal lattice.  These defects often directly or indirectly control the material properties such as luminescence efficiency and carrier lifetime.  In addition, presence of defects may lead to formation of midgap states which causes n-type conductivity and  visible emissions other than the band to band emission. Many experiments have been devoted to characterize these native defects and impurities, in the hope that it will improve the properties of ZnO to its intended application. Thus, understanding the behavior of defects in ZnO crystal is important to its successful application as a semiconductor.  

First principles or ab-initio calculations have offered various insights into the nature of defects without the need for experimentation. Density Functional Theory (DFT) have been used in most first-principles calculations which proved to be an indispensable tool in probing the energetics, atomic and electronic structures of defects in semiconductors, including  but not limited to native defects, impurities, dopants, and surface defects. 

There are many published papers on first-principles  calculations based on DFT with exchange-correlation functionals of Local Density Approximation (LDA) and Generalized Gradient Approximation (GGA) for native point defects in wurtzite ZnO. However, standard DFT calculations have led to severe underestimation of the  band gap which results to uncertainties in the formation energies and transition levels of the defect states.  These uncertainties have led to significant discrepancies in the conclusions drawn from various published reports using such calculations. Hence, various correction schemes have been employed to improve the band gap. One such corrective approach is the use of Hubbard-U method  to treat the over-delocalization of $3d$ \ch{Zn} orbitals and $2p$ \ch{O} orbitals,  thereby increasing the band gap by shifting the valence band downwards and conduction band upwards. 

\section{Purpose and Motivation}
Describe the importance of defects in ZnO
\section{Objectives}
\begin{itemize}
    \item to obtain the bandstructure and density of states of native point defects in wurtzite \ch{ZnO}
    \item to determine which atomic orbitals  contribute to the defect energy level
\end{itemize}

This study considers only the native point defects in ZnO: oxygen and zinc vacancies (\ch{V_O} and \ch{V_{Zn}}), interstitials (\ch{O_i} and \ch{Zn_i}), antisites (\ch{O_{Zn}} and \ch{Zn_O}) and charged defects (\ch{V_O^{1+}}, \ch{V_O^{2+}}, \ch{Zn_i^{1+}} and \ch{Zn_i^{2+}}). 
% Study the mechanisms of different defects in ZnO
\section{Outline}

% \chapter{Acronym Used}
