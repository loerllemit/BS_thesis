\addchapheadtotoc

\chapter{Introduction}
Zinc Oxide (ZnO) has remained the focus of rigorous theoretical and experimental studies owing to its unique electronic and optical properties, and its application as an energy sources \citep{Look2001, Oezguer2005, Klingshirn2007, Janotti2009, KolodziejczakRadzimska2014}. ZnO is a wide direct band gap with an experimental value ranging from 3.44 eV at 4 K to  3.37 eV at 300 K emitting at ultraviolet region \citep{Reynolds1999,Camarda2016,Chitra2020}. In addition, this ionic semiconductor is known to host excitons with a large binding energy of 60 meV that persists even above room temperature \citep{Oezguer2005}, rendering its utility as a UV laser source \citep{Bagnall1997}, tunable UV photodetectors \citep{Rodnyi2011}, and scintillator \citep{Gorokhova2008,Ji2016}.

However, realization of a perfect crystal is very difficult or impossible to achieve in experimental conditions due to presence of impurities even at ultra-high vacuum environment \citep{Konishi1996} and possibility of forming native defects due to imperfections in the crystal lattice \citep{Janotti2007}.  These defects often directly or indirectly control the material properties such as luminescence efficiency and carrier lifetime \citep{Sharma2010}.  In addition, presence of defects may lead to formation of midgap states which causes visible emissions other than the band to band emission \citep{Wang2015}. The nonstoichiometry as well as unintenional n-type conductivity in ZnO were also attributed to defects \citep{Oba2011,Liu2016}.  Many experiments have been devoted to characterize these native defects and impurities, in the hope that it will improve the properties of ZnO to its intended application \citep{Agulto2018,Musavi2019,Rudra2020,Chen2020,Bian2020}. Thus, understanding the behavior of defects in ZnO crystal is important to its successful application as a semiconductor.

First principles or ab-initio calculations have offered various insights into the nature of defects without the need for experimentation. Density Functional Theory (DFT) \citep{Hohenberg1964,Kohn1965} have been used in most first-principles calculations which proved to be an indispensable tool in probing the energetics, atomic and electronic structures of defects in semiconductors, including  but not limited to native defects, impurities, dopants, and surface defects \citep{Yang2011,Masoumi2018,Vrubel2020}.

There are many published papers on first-principles  calculations based on DFT with exchange-correlation functionals of Local Density Approximation (LDA) and Generalized Gradient Approximation (GGA) for native point defects in wurtzite ZnO \citep{Kohan2000,Walle2001,Zhang2001,Oba2001,Lee2001,Erhart2005,Erhart2006,Lany2007}. However, standard DFT calculations have led to severe underestimation of the  band gap which results to uncertainties in the formation energies and transition levels of the defect states.  These uncertainties have led to significant discrepancies in the conclusions drawn from various published reports using such calculations. Hence, various correction schemes have been employed to improve the band gap. One such corrective approach is the use of Hubbard-U method  to treat the over-delocalization of $3d$ \ch{Zn} orbitals and $2p$ \ch{O} orbitals,  thereby increasing the band gap by shifting the valence band downwards and conduction band upwards \citep{Qiao2014,Yaakob2014,Parhizgar2018,Harun2020}.

\section{Purpose and Motivation}
In order to characterize and validate the effectiveness of ZnO as a potential semiconductor in its own niche of applications, it is important to first gain knowledge of the energetics of  defects formation. The properties of semiconductors in general are dependent on the processes that are occurring on the atomic level inside the solid crystal. Introducing defects such as vacancies, interstitials, substitutions, antisites, and impurity atoms cause the formation of mid-gap states. These states are populated by charge carriers and serve as luminescent centers, thereby producing emissions in the visible region. It is therefore imperative to know the underlying mechanisms of these defects so that it can be possible to tune its properties according to its intended purpose. First principle calculations, in particular Density Functional Theory (DFT), will enable first order exploration on the characteristics of these defects without the need of doing physical experimentation. Investigations become more focused as soon as simulations indicate most likely pathway. It is very useful to have \emph{a priori} knowledge of defects in order to optimize the appropriate parameters used in the experiment, thereby saving  manhours and materials consumption.

\section{Objectives}
The main goal of this thesis is to understand the physics  of defect formation through the study of its energetics and electronic structures. Specifically, the study aims to

\begin{itemize}
	\item obtain the band structure and density of states of native point defects in wurtzite \ch{ZnO}
	\item determine which atomic orbitals  contribute to the defect energy level through projected density of states calculation
	\item calculate the defect formation energies and quantitatively describe the stabilities of defects
	\item calculate the transition level of charged defects
\end{itemize}

\section{Scope and Limitations}
This study considers only the native point defects in ZnO: oxygen and zinc vacancies (\ch{V_O} and \ch{V_{Zn}}), interstitials (\ch{O_i} and \ch{Zn_i}), antisites (\ch{O_{Zn}} and \ch{Zn_O}) and charged defects (\ch{V_O^{1+}} and \ch{V_O^{2+}}; \ch{V_{Zn}^{1-}} and \ch{V_{Zn}^{2-}}; \ch{O_i^{1-}} and \ch{O_i^{2-}}; \ch{Zn_i^{1+}} and \ch{Zn_i^{2+}}). Doping of impurity atoms are not considered in this study. Surface states that exist in vacuum-surface boundaries are not considered also since the study is limited to simulation of bulk solids. The correction scheme used for underestimation of band gap is Hubbard-U method and no additional corrections to address spurious interactions between charged defects since this study is not rigorous enough to investigate these interactions.


\section{Thesis Outline}
This thesis is organized as follows. Chapter \ref{chap:rrl} will discuss related literatures about ZnO. This includes pertinent properties of ZnO such as crystal structure and native point defects. Chapter \ref{chap:theory} will discuss the theoretical framework in doing Density Functional Theory (DFT) calculations. Included in this chapter are various approximations used to simplify calculations of many-body problem into an independent non-interacting body problem. Chapter \ref{chap:compute} will discuss the application of DFT in bulk solids. The bulk solid will be modeled based on supercell approach and necessary data sampling techniques needed to have a computationally tractable simulation are discussed in this chapter. Chapter \ref{chap:software} will discuss the technical details of running DFT in a simulation software. Chapter \ref{chap:rnd} will focus on the results of the simulation of the native point defects in ZnO. This chapter will report the energetics and electronic structures of the defects. Lastly, Chapter \ref{chap:conclu} will the summarize the results and will offer some recommendations.
