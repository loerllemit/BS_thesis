\chapter{DFT Calculation of Solids}
\section{Basis Sets}
Solving the Kohn-Sham equation in \eqref{eq:KS} requires the use of mathematical representations to describe the single-particle Kohn-Sham orbitals $\phi_i(\va*{r})$. One possibility is to express these orbitals as a basis sets that are known and numerically solvable. One starts by expressing orbitals as a linear combination of generic basis set 

\begin{equation} \label{eq:basis_set}
    \phi_i(\va*{r}) = \sum^M_\alpha c^i_\alpha \ket{\chi_\alpha}
\end{equation}
where $i$ is the band index, the sum runs over all the basis functions up to the dimension $M$, and $c^i_\alpha$ is the expansion coefficient of a known basis function $\ket{\chi_\alpha}$. Since $\phi_i(\va*{r})$ spans the whole infinite space, $M$ must be in principle infinite. However, in practice the basis set is truncated just enough for an accurate description of the orbital. The choice of the basis set depends on several factors such as (a) efficiency and (b) unbiased \citep{Cottenier2013}. A basis set is efficient if it resembles  $\phi_i(\va*{r})$ closely, hence requiring less expansion coefficients and smaller dimension size. However, this assumes that the solution to the problem must be known beforehand. Such basis set can never be general because it will quickly yield a solution for a specific problem but will poorly perform for other cases. The problem is that optimizing a basis set for a specific system can cause bias. This means that if a property of a system is calculated, but the basis set is optimized for only one particular system, the result will be biased towards that one \citep{Junquera2001,Louwerse2012}. It is the goal of theoretical condensed matter physics to find a basis set that is simultaneously efficient and unbiased. There are three types of basis sets that are commonly used for expansions, namely: local, nonlocal, and augmented basis. 


\subsection{Local Basis Set}
A local basis has its peak centered on a local point and is well applicable to orbitals around individual atoms in real space. Gaussian basis sets or any atom-centered basis orbitals are examples of this type. It is a popular choice for atoms and molecules whose orbitals are highly localized around each atom.


Since it is local, this type of basis cannot properly described the long-range interaction of metals and the periodicity of crystalline solids. 

\subsection{Nonlocal Basis Set}
Plane Wave
\subsection{Augmented Basis Set}
\subsection{Gaussian Orbital}

    \subsection{Slater type orbitals}
    \subsection{Plane Wave}
    \subsection{Augmented Plane Wave}

\section{Matrix Formulation for KS equation}
The use of basis sets transforms the Kohn-Sham equation into an ordinary matrix algebra  that can be solved numerically. The Kohn-Sham equation in \eqref{eq:KS} is expanded in terms of basis sets using \eqref{eq:basis_set} 

\begin{equation}\label{eq:KS_basis}
	\hat{\mathcal{H}}_{KS}\,  \sum^M_\alpha c^i_\alpha \ket{\chi_\alpha} = \epsilon_i\,  \sum^M_\alpha c^i_\alpha \ket{\chi_\alpha}
\end{equation}
Left "multiply" with  $\bra{\chi_\beta}$:
\begin{equation}
    \sum^M_\alpha  \bra{\chi_\beta} \hat{\mathcal{H}}_{KS} \ket{\chi_\alpha} c^i_\alpha = \sum^M_\alpha \braket{\chi_\beta}{\chi_\alpha} c^i_\alpha \epsilon_i
\end{equation}
which can be simplified as 
\begin{equation}
    \sum^M_\alpha  H_{\beta\alpha} c^i_\alpha = \sum^M_\alpha S_{\beta\alpha} c^i_\alpha \epsilon_i
\end{equation}
where $ H_{\beta\alpha}$ and $S_{\beta\alpha}$ are the energy-independent Hamiltionian and the overlap matrix, respectively \citep{Kohanoff2006}. The elements of these matrices are defined as 

\begin{align}
    H_{\beta\alpha} &= \bra{\chi_\beta} \hat{\mathcal{H}}_{KS} \ket{\chi_\alpha} = \int \chi^*_\beta(\va*{r}) \hat{\mathcal{H}}_{KS}\chi_\alpha(\va*{r}) \dd[3]{r}\\
    S_{\beta\alpha} &= \braket{\chi_\beta}{\chi_\alpha} = \int \chi^*_\beta(\va*{r}) \chi_\alpha(\va*{r}) \dd[3]{r}
\end{align}
The overlap matrix $S_{\beta\alpha}$ takes into account the possible non-orthogonality of the basis functions \citep{Woods2019}. The general matrix eigenvalue problem can be recast into a compact form \citep{Requist2008}
\begin{equation} \label{eq:matrix_eigen}
    \bm{H c =  S c \Lambda}
\end{equation}
where $\bm{\Lambda}$ is the diagonal matrix containing energy eigenvalues and  $\bm{c}$ has the eigenfunction (expansion coefficients of the KS orbital) as columns. In solving \eqref{eq:matrix_eigen}, the normalization condition must be taken into account
\begin{align} 
   \int \phi_i(\va*{r}) \phi^*_i(\va*{r}) &= \int \sum^M_\alpha \sum^M_\beta c^{i*}_\alpha \chi^*_\alpha c^i_\beta \chi_\beta \dd[3]{r} = 1 \\
   &= \sum^M_\alpha \sum^M_\beta c^{i*}_\alpha c^i_\beta \int \chi^*_\alpha  \chi_\beta\, \dd[3]{r} = 1 \\
   &= \sum^M_\alpha \sum^M_\beta c^{i*}_\alpha c^i_\beta S_{\alpha\beta} = 1 \label{eq:orbital_norm}
\end{align}

There are $M \times N$ elements of $\bm{c}$ needed to be solved, where $M$ is the total number of basis functions used and $N$ is the total number of lowest-energy orbitals. In addition, there are $N$ unknown energy eigenvalues to be solved. Fortunately, there are $M \times N$ independent equations in \eqref{eq:matrix_eigen} and $N$ equations coming from the normalization condition \eqref{eq:orbital_norm} so that $N(M+1)$ simultaneous equations are solved \citep{Cottenier2013}. It is obvious that increasing either $M$ or $N$ will increase the computational power needed. This does not inlude yet the  iterative self consistent field calculation, as shown in Figure \ref{fig:KS_loop}, needed to have converged electron density. 

%kohanoff

\newpage

\section{Pseudopotential Approach}
    This is sample text
    \subsection{Freezing the core electrons}
    \subsection{Pseudizing the valence electrons}
    \subsection{Common Pseudopotentials}
        \subsubsection{Norm-Conserving PP}
        \subsubsection{Ultrasoft PP}
        \subsubsection{Projector Augmented Wave}
\section{Choosing the appropriate Calculation Size}
    \subsection{Use of Supercell}
        \subsubsection{Periodic Boundary Conditions (PBC)}
    \subsection{Use of Reciprocal Space}
        \subsubsection{Reciprocal Lattice}
        \subsubsection{First Brillouin Zone}
        \subsubsection{Irreducible Brillouin Zone}
    \subsection{k-point sampling}
        \subsubsection{Monkhorst-Pack method}
        \subsubsection{Gamma Point Sampling}
        Example of double quotes ``word''. Lore
\section{Bloch Representations}
    \subsection{Electrons in solid}
    \subsection{Bloch Theorem in periodic systems}
    \subsection{Fourier Expansion of Bloch representations}
        \subsubsection{Fourier Expansions}
        \subsubsection{Fast Fourier Transformation (FFT)}
        \subsubsection{Kohn-Sham Matrix Representations}
\section{Plane Wave (PW) Expansion}
    \subsection{Basis Set}
        \subsubsection{Local Basis Set}
        \subsubsection{Plane Wave Basis Set}
    \subsection{Plane Wave Expansion for KS quantities}
        \subsubsection{Charge Density}
        \subsubsection{Kinetic Energy}
        \subsubsection{Effective Potential}
\section{Electronic Structure}
    \subsection{Band Structure of free electrons}
    \subsection{Band Structure of electrons in solids}
    \subsection{Electronic Density of States}

    HELLO
\section{Practical Aspects}
    \subsection{Relaxation}
    TEST FILES
    \subsection{Energy Cutoffs}
    HELLOW WORLD
        \subsubsection{Cutoff for Wavefunction}
        \subsubsection{Cutoff for Charge Density}
    \subsection{Smearing}
        \subsubsection{Gaussian Smearing}
        \subsubsection{Fermi Smearing}
        \subsubsection{Methfessel–Paxton Smearing}





