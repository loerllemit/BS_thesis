\chapter{DFT Calculation of Solids}
\section{Basis Sets}
Solving the Kohn-Sham equation in \eqref{eq:KS} requires the use of mathematical representations to describe the single-particle Kohn-Sham orbitals $\phi_i(\va*{r})$. One possibility is to express these orbitals as a basis sets that are known and numerically solvable. One starts by expressing orbitals as a linear combination of generic basis set 

\begin{equation} \label{eq:basis_set}
    \phi_i(\va*{r}) = \sum^M_\alpha c^i_\alpha \ket{\chi_\alpha}
\end{equation}
where $i$ is the band index, the sum runs over all the basis functions up to the dimension $M$, and $c^i_\alpha$ is the expansion coefficient of a known basis function $\ket{\chi_\alpha}$. Since $\phi_i(\va*{r})$ spans the whole infinite space, $M$ must be in principle infinite. However, in practice the basis set is truncated just enough for an accurate description of the orbital. The choice of the basis set depends on several factors such as (a) efficiency and (b) unbiased \citep{Cottenier2013}. A basis set is efficient if it resembles  $\phi_i(\va*{r})$ closely, hence requiring less expansion coefficients and smaller dimension size. However, this assumes that the solution to the problem must be known beforehand. Such basis set can never be general because it will quickly yield a solution for a specific problem but will poorly perform for other cases. The problem is that optimizing a basis set for a specific system can cause bias. This means that if a property of a system is calculated, but the basis set is optimized for only one particular system, the result will be biased towards that one \citep{Junquera2001,Louwerse2012}. It is the goal of theoretical condensed matter physics to find a basis set that is simultaneously efficient and unbiased. There are three types of basis sets that are commonly used for expansions, namely: local, nonlocal, and augmented basis. 


\subsection{Local Basis Set}
A local basis has its peak centered on a local point and is well applicable to orbitals around individual atoms in real space. Gaussian basis sets or any atom-centered basis orbitals are examples of this type. It is a popular choice for atoms and molecules whose orbitals are highly localized around each atom.
Hence, less than 20 basis functions per atom are sufficient enough to achieve acceptable accuracy. As an example, the Slater type basis orbitals (STO) are written as \citep{Slater1932}
\begin{equation}
    \ket{\chi^{\text{STO}}_\alpha} = A e^{-\alpha r}
\end{equation}
where $A$ is some normalization constant. Note that STO exponentially decays away from an atom centered at $\va*{r}$. On the other hand, Gaussian type basis orbitals (GTO)  are written as \citep{Boys1950}
\begin{equation}
    \ket{\chi^{\text{GTO}}_\alpha} = A e^{-\alpha r^2}
\end{equation}
GTO has the advantage that all integrals associated with it can be performed analytically. Since these basis sets are localized, they cannot properly described the long-range interaction of metals and the periodicity of crystalline solids. 

\subsection{Nonlocal Basis Set}
Nonlocal basis set span the whole space. An important class of basis orbital under this category is the plane wave basis (PW) described as 
\begin{equation}
    \ket{\chi^{\text{PW}}_\alpha} = A e^{i \alpha r}
\end{equation}
which can be generalized as a dot product of wavevector $\va{k}$ and position vector $\va{r}$
\begin{equation}
    \ket{\chi^{\text{PW}}_{\va{k}}} = A e^{i \va{k} \vdot \va{r}}
\end{equation}
PWs are the most commonly used in DFT  because of the following reasons: PWs are already solutions to periodic systems satisfying the Bloch condition; PWs are convenient  in taking gradients and integrals because of their exponential form; changing the domain of PWs from  real space to reciprocal space are easily executed using Fourier transformation; PWs are orthogonal which simplifies calculation; and lastly, PWs are independent of the atomic positions because of its nonlocal nature. However, there are also disadvantages of using it. It requires an enormous amount of PWs to properly describe the rapid fluctuation of orbital wavefunctions near the core region of an atom or ion. A direct fix to this problem is the application of pseudopotentials to smoothen the strong Coulomb potential of the nucleus, which will be the topic of later section \citep{Segall2002}. 

\subsection{Augmented Basis Set}
Augmented basis sets are combinations of local and nonlocal basis sets. Under this category is the Augmented plane waves (APW) basis. The APW divides the space into two regions: the core region,where the orbitals are atomic-like; and the interstital region, where the orbitals  resemble plane waves \citep{Slater1953}. The basis orbitals are taken to be \citep{Sjoestedt2000}
\begin{equation}
    \ket{\chi^{\text{APW}}_{\va{k}}} = 
    \begin{dcases}
        \text{atomic basis}    \quad &, \abs{\va{r}-\va{R}} \leq r_c \\
        A e^{i \va{k} \vdot \va{r}} &, \abs{\va{r}-\va{R}} > r_c \\
    \end{dcases}
\end{equation}
where $\va{R}$ is the center of the atom and $r_c$ is the core radius. Outside the core, the orbital wavefunction is a plane wave because the potential is constant there. Inside the core, the orbital wavefunction is atomic-like and can be solved by the appropriate Schr\"{o}dinger equation. The potential involved in this type of basis is usually called muffin-tin potential due its resemblance to muffin tins. APWs must satisfy the boundary conditions at $\abs{\va{r}-\va{R}} = r_0$. That is, the basis orbital must be continuous at the boundary value and its slope exists \citep{Andersen1975}. Augmented plane waves are very accurate because it describes both core electrons and valence electrons well. However, accuracy is always associated with computational costs. 



\section{Matrix Formulation for KS equation}
The use of basis sets transforms the Kohn-Sham equation into an ordinary matrix algebra  that can be solved numerically. The Kohn-Sham equation in \eqref{eq:KS} is expanded in terms of basis sets using \eqref{eq:basis_set} 

\begin{equation}\label{eq:KS_basis}
	\hat{\mathcal{H}}_{KS}\,  \sum^M_\alpha c^i_\alpha \ket{\chi_\alpha} = \epsilon_i\,  \sum^M_\alpha c^i_\alpha \ket{\chi_\alpha}
\end{equation}
Left "multiply" with  $\bra{\chi_\beta}$:
\begin{equation}
    \sum^M_\alpha  \bra{\chi_\beta} \hat{\mathcal{H}}_{KS} \ket{\chi_\alpha} c^i_\alpha = \sum^M_\alpha \braket{\chi_\beta}{\chi_\alpha} c^i_\alpha \epsilon_i
\end{equation}
which can be simplified as 
\begin{equation}
    \sum^M_\alpha  H_{\beta\alpha} c^i_\alpha = \sum^M_\alpha S_{\beta\alpha} c^i_\alpha \epsilon_i
\end{equation}
where $ H_{\beta\alpha}$ and $S_{\beta\alpha}$ are the energy-independent Hamiltionian and the overlap matrix, respectively \citep{Kohanoff2006}. The elements of these matrices are defined as 

\begin{align}
    H_{\beta\alpha} &= \bra{\chi_\beta} \hat{\mathcal{H}}_{KS} \ket{\chi_\alpha} = \int \chi^*_\beta(\va*{r}) \hat{\mathcal{H}}_{KS}\chi_\alpha(\va*{r}) \dd[3]{r}\\
    S_{\beta\alpha} &= \braket{\chi_\beta}{\chi_\alpha} = \int \chi^*_\beta(\va*{r}) \chi_\alpha(\va*{r}) \dd[3]{r}
\end{align}
The overlap matrix $S_{\beta\alpha}$ takes into account the possible non-orthogonality of the basis functions \citep{Woods2019}. Note that for plane wave (PW) basis sets, which are orthonormal, the overlap matrix $S_{\beta\alpha}$ becomes an unit matrix. The general matrix eigenvalue problem can be recast into a compact form \citep{Requist2008}
\begin{equation} \label{eq:matrix_eigen}
    \bm{H c =  S c \Lambda}
\end{equation}
where $\bm{\Lambda}$ is the diagonal matrix containing energy eigenvalues and  $\bm{c}$ has the eigenfunction (expansion coefficients of the KS orbital) as columns. In solving \eqref{eq:matrix_eigen}, the normalization condition must be taken into account
\begin{align} 
   \int \phi_i(\va*{r}) \phi^*_i(\va*{r}) &= \int \sum^M_\alpha \sum^M_\beta c^{i*}_\alpha \chi^*_\alpha c^i_\beta \chi_\beta \dd[3]{r} = 1 \\
   &= \sum^M_\alpha \sum^M_\beta c^{i*}_\alpha c^i_\beta \int \chi^*_\alpha  \chi_\beta\, \dd[3]{r} = 1 \\
   &= \sum^M_\alpha \sum^M_\beta c^{i*}_\alpha c^i_\beta S_{\alpha\beta} = 1 \label{eq:orbital_norm}
\end{align}

There are $M \times N$ elements of $\bm{c}$ needed to be solved, where $M$ is the total number of basis functions used and $N$ is the total number of lowest-energy orbitals. In addition, there are $N$ unknown energy eigenvalues to be solved. Fortunately, there are $M \times N$ independent equations in \eqref{eq:matrix_eigen} and $N$ equations coming from the normalization condition \eqref{eq:orbital_norm} so that $N(M+1)$ simultaneous equations are solved \citep{Cottenier2013}. It is obvious that increasing either $M$ or $N$ will increase the computational power needed. This does not inlude yet the  iterative self consistent field calculation, as shown in Figure \ref{fig:KS_loop}, needed to have converged electron density. Also, most DFT codes spend substantial amount of time in matrix diagonalization.

\section{Pseudopotential (PP) Approach}
    The idea behind the use of pseudopotentials is to replace the strong Coulomb potential of the nucleus by an effective potential acting on the valence electrons \citep{Phillips1958,Phillips1959,Cohen1970}. When atoms bond together to form a solid, the core electrons are so localized in a deep potential well that they remain invariant. Thus, their contribution to bonding is negligible and the replacement of its potential by a simple fictitious potential is justified. Figure \ref{fig:PP} illustrates the action of pseudopotential on the wavefunction and potential of an atomic orbital. The all-electron wavefunction contains nodes which are  computationally difficult to  solve. On the other hand, pseudo wavefunction is nodeless everywhere, and therefore  it greatly reduces the number of plane waves required for the calculation by a significant amount. Note that at large distances away from the nucleus, both potential becomes constant and the wavefunction is expected to be a plane wave. By effectively neglecting the core electrons from the calculation, the Kohn-Sham orbitals needed is dramatically reduced. This will substantially reduce the computational time required to calculate orbital-dependent quantities.

    \begin{figure}[tbh!]
        \centering
        \includegraphics[width=0.8\linewidth]{"images/computational/PP_combine"}
        \caption[ Schematic illustration of a pseudo wavefunction pseudized from a 3s wavefunction of Si orbital]{Schematic illustration of a  (a) pseudo wavefunction pseudized from a 3s wavefunction of Si orbital and the (b) corresponding pseudo- (PP) and all-electron (AE) potentials. The all-electron approach takes into account all electrons including core and valence electrons. Illustration taken from \citep{Jochym2008}.}
        \label{fig:PP}
    \end{figure}

There are two criteria for choosing a good pseudopotential, namely: softness and transferability \citep{Troullier1990,Fuchs1999}. A pseudopotential is soft  if it requires few plane waves to model the system. This is similar to efficiency of basis sets. A pseudopotential is transferable  if it can be used in whatever environment (e.g. molecule, solid, cluster, surface, metal, insulator, etc). The choice depends on which pseudopotential is advantageous to use  and the type of calculation being done. The common pseudopotentials used in DFT codes are Norm-Conserving pseudopotential, Ultrasoft pseudopotential, and Projector Augmented Wave. 

\subsection{Norm-Conserving Pseudopotential (NCPP)}
In norm-conserving pseudopotentials, the pseudopotential and all-electron charge densities are set  equal so that the norm is conserved in both potentials \citep{Hamann1979,Troullier1991}. Pseudopotentials are generated to meet this criterion 
\begin{equation}
    \int_0^{r_c} \abs{\phi_{\text{PP}}(\va*{r})}^2 \dd{\va*{r}} = \int_0^{r_c} \abs{\phi_{\text{AE}}(\va*{r})}^2 \dd{\va*{r}}
\end{equation} 
where $r_c$ is the chosen cutoff radius that separates the core region from the valence region. The constraint imposed on this pseudopotential leads to an improvement in transferability of potentials to different chemical environments. In addition, reducing $r_c$ improves the transferability  because in this way the pseudo wavefunction becomes closer to the all-electron result. However, the cutoff radius should be chosen outside the location of the maximum node of the all-electron wavefunction. Note that this pseudopotential gives only the valence charge density and not the total charge density. Other norm-conserving schemes were proposed by Troullier and Martin (TM) \citep{Troullier1991}, and by Rappe, Rabe, Kaxiras, and Joannopoulos (RRKJ) \citep{Rappe1990}.

\subsection{Ultrasoft Pseudopotential (USPP)}
Ultrasoft pseudopotentials were introduced in order for the calculations to have lowest numbers of plane waves basis set used since it was shown that the norm of the all-electron and pseudo wavefunction was not necessary requirement for transferability. Hence, this was done by Vanderbilt \citep{Vanderbilt1990} who showed that smoother but highly transferable pseudopotentials are possible. The cutoff radius $r_c$ is situated farther than the equivalent norm-conserving pseudopotential and the pseudo wavefunction is flatter. This leads to fewer plane waves that gives significant reduction in computational time. Similar to norm-conserving pseudopotentials, the ultrasoft pseudopotential only gives valence charge densities, not total charge
densities.

\subsection{Projector Augmented Wave (PAW)}
The Projector Augmented Wave takes into account both all-electron and pseudo wavefunction into calculations. It aims 
for both efficiency of using pseudopotential and the accuracy of using all-electron potential \citep{Bloechl1994,Kresse1999}. However, the all-electron wavefunction is limited only on the core region and will be truncated beyond the cut-off radius $r_c$. A correction factor is added to subtract the overlapping part of the pseudo wavefunction in the core region \citep{Rostgaard2009}. Hence, the PAW wavefunction involves three terms
\begin{equation}
    \psi_{\text{PAW}} =  \psi_{\text{AE}} + \psi_{\text{PP}}  - \psi_{\text{net}}
\end{equation}
The actions of the terms in the equation above are visualized in Figure \refeq{fig:PAW}. The $\psi_{\text{PP}}$ is expanded in plane wave basis sets while $\psi_{\text{AE}}$ is only defined within the cutoff radius $r_c$. The $\psi_{\text{net}}$ subtracts the overlapping part of $\psi_{\text{PP}}$ in the core region. 

Note that in Kohn-Sham formulation, these wavefunctions $\psi$ become the independent Kohn-Sham orbitals $\phi$. PAW calculations are accurate as all-electron calculations with much less computational effort. Unlike the two pseudopotentials mentioned before, PAW pseudopotential returns both the core and valence charge densities. 

\begin{figure}[tbh!]
    \centering
    \includegraphics[width=0.48\linewidth]{"images/computational/PAW"}
    \caption[ Schematic illustration of the wavefunctions used in PAW pseudopotential]{Schematic illustration of the wavefunctions used in PAW pseudopotential. Illustration taken from \citep{Lee2016}.}
    \label{fig:PAW}
\end{figure}

\section{Supercells}
Most solids are characterized by its regular repeating three-dimensional structure called a crystal lattice. Hence, it is possible to study solids by just looking at the building block, which is ordinarily called the unit cell. In order to model solids that are feasible for computational simulation, repeating unit cells that are stack together must be needed. These  stacked unit cells  are collectively called supercell. When implementing DFT, the periodic boundary conditions must be taken into account. In this case, the supercell is duplicated periodically throughout the whole space. However, the actual calculation is applied only on a single supercell while the rest (called images) simply copies it with no significant computational cost. 

When defects are introduced into the supercell, it forms a periodic array of defects across all the images of the supercell. A supercell must be large enough so that the calculation is independent of the location of the defect inside the supercell and also to reduce  the interaction  with its images. Thus, the Kohn-Sham equation and other pertinent calculations are solved only within  a single supercell.

%add figure

\section{DFT Calculation in Reciprocal Space}
In previous discussions,  $k$ was defined as any wavevector in the reciprocal space. However, it can always be transform to $k \rightarrow k + G$ so that the new $k$ is in the first Brillouin zone and that any wavevectors are equivalent to the new one by a reciprocal lattice vector $G$. See Appendix \ref{chap:BZ} for discussions about reciprocal lattice and Brillouin zones. This transformation will limit DFT calculations inside the first Brillouin zone instead of the whole reciprocal space. 


\section{Choosing the appropriate Calculation Size}
    \subsection{Use of Supercell}
        \subsubsection{Periodic Boundary Conditions (PBC)}
    \subsection{Use of Reciprocal Space}
        \subsubsection{Reciprocal Lattice}
        \subsubsection{First Brillouin Zone}
        \subsubsection{Irreducible Brillouin Zone}
    \subsection{k-point sampling}
        \subsubsection{Monkhorst-Pack method}
        \subsubsection{Gamma Point Sampling}
        Example of double quotes ``word''. Lore
\section{Bloch Representations}
    \subsection{Electrons in solid}
    \subsection{Bloch Theorem in periodic systems}
    \subsection{Fourier Expansion of Bloch representations}
        \subsubsection{Fourier Expansions}
        \subsubsection{Fast Fourier Transformation (FFT)}
        \subsubsection{Kohn-Sham Matrix Representations}
\section{Plane Wave (PW) Expansion}
    \subsection{Basis Set}
        \subsubsection{Local Basis Set}
        \subsubsection{Plane Wave Basis Set}
    \subsection{Plane Wave Expansion for KS quantities}
        \subsubsection{Charge Density}
        \subsubsection{Kinetic Energy}
        \subsubsection{Effective Potential}
\section{Electronic Structure}
    \subsection{Band Structure of free electrons}
    \subsection{Band Structure of electrons in solids}
    \subsection{Electronic Density of States}

    HELLO
\section{Practical Aspects}
    \subsection{Relaxation}
    TEST FILES
    \subsection{Energy Cutoffs}
    HELLOW WORLD
        \subsubsection{Cutoff for Wavefunction}
        \subsubsection{Cutoff for Charge Density}
    \subsection{Smearing}
        \subsubsection{Gaussian Smearing}
        \subsubsection{Fermi Smearing}
        \subsubsection{Methfessel–Paxton Smearing}





