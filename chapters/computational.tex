\chapter{DFT Calculation of Solids}
\section{Basis Sets}
    \subsection{Plane Wave}
    \subsection{Gaussian Orbital}
    \subsection{Slater type orbitals}
\section{Pseudopotential Approach}
    This is sample text
    \subsection{Freezing the core electrons}
    \subsection{Pseudizing the valence electrons}
    \subsection{Common Pseudopotentials}
        \subsubsection{Norm-Conserving PP}
        \subsubsection{Ultrasoft PP}
        \subsubsection{Projector Augmented Wave}
\section{Choosing the appropriate Calculation Size}
    \subsection{Use of Supercell}
        \subsubsection{Periodic Boundary Conditions (PBC)}
    \subsection{Use of Reciprocal Space}
        \subsubsection{Reciprocal Lattice}
        \subsubsection{First Brillouin Zone}
        \subsubsection{Irreducible Brillouin Zone}
    \subsection{k-point sampling}
        \subsubsection{Monkhorst-Pack method}
        \subsubsection{Gamma Point Sampling}
        Example of double quotes ``word''. Lore
\section{Bloch Representations}
    \subsection{Electrons in solid}
    \subsection{Bloch Theorem in periodic systems}
    \subsection{Fourier Expansion of Bloch representations}
        \subsubsection{Fourier Expansions}
        \subsubsection{Fast Fourier Transformation (FFT)}
        \subsubsection{Kohn-Sham Matrix Representations}
\section{Plane Wave (PW) Expansion}
    \subsection{Basis Set}
        \subsubsection{Local Basis Set}
        \subsubsection{Plane Wave Basis Set}
    \subsection{Plane Wave Expansion for KS quantities}
        \subsubsection{Charge Density}
        \subsubsection{Kinetic Energy}
        \subsubsection{Effective Potential}
\section{Electronic Structure}
    \subsection{Band Structure of free electrons}
    \subsection{Band Structure of electrons in solids}
    \subsection{Electronic Density of States}

    HELLO
\section{Practical Aspects}
    \subsection{Relaxation}
    TEST FILES
    \subsection{Energy Cutoffs}
    HELLOW WORLD
        \subsubsection{Cutoff for Wavefunction}
        \subsubsection{Cutoff for Charge Density}
    \subsection{Smearing}
        \subsubsection{Gaussian Smearing}
        \subsubsection{Fermi Smearing}
        \subsubsection{Methfessel–Paxton Smearing}





