\chapter{Review of Related Literature} \label{chap:rrl}
% \section{Semiconductors}
%     \subsection{Properties}
%     \subsection{Applications of Semiconductors}
%     \subsection{Defects in Semiconductors}
\section{Properties of ZnO}
% \section{Zinc Oxide}
	describe ZnO in broad perspective
	donors and acceptors
	excitation in semiconductors
\section{Crystal Structure}
There are three possible crystal structures (phases) of \ch{ZnO}, namely: wurtzite, zinc blende, and rocksalt. The wurtzite structure belongs to the space group $C^4_{6v}$ in Schoenflies notation, $P6_3mc$ in the  Hermann–Mauguin notation, and space group number 186 in International Tables for Crystallography (ITA) notation \citep{Hahn2005}. The crystal structure of wurtzite ZnO is shown in Figure \ref{fig:ZnO_unit}. Each zinc atom is surrounded by four oxygen atoms, which are located
at the corners of an irregular tetrahedron.

\begin{figure}[tbh!] 
	\centering
	\includegraphics[width=0.5\linewidth]{"images/rrl/ZnO_unit"}
	\caption[Crystal structure of wurtzite \ch{ZnO} unit cell]{Crystal structure of wurtzite \ch{ZnO} unit cell}
	\label{fig:ZnO_unit}
\end{figure}

\section{Brillouin Zone Symmetry}
	The primitive vectors of the direct (real) hexagonal lattice are 
\begin{align}
	\va{a}_1 &= a \vu*{x}	\\
	\va{a}_2 &= \frac{a}{2} \vu*{x} + \frac{\sqrt{3}a}{2} \vu*{y} \\
	\va{a}_3 &= c \vu*{z}
\end{align}
where $a$ and $c$ are the lattice parameters of hexagonal \ch{ZnO}. On the other hand, the primitive vectors of the reciprocal lattice can be derived using the formula \eqref{eq:recipro_real} in Appendix \ref{chap:BZ} 

\begin{align}
	\va{b}_1 &= \frac{2 \pi}{a} \left(\vu*{k}_x - \frac{1}{\sqrt{3}} \vu*{k}_y \right) \\ 
	\va{b}_2 &= \frac{2 \pi}{a} \left( \frac{2}{\sqrt{3}} \vu*{k}_y \right) \\
	\va{b}_3 &= \frac{2 \pi}{a} \left( \frac{a}{c} \vu*{k}_z  \right)
\end{align}

The first Brillouin Zone of an hexagonal lattice is also an hexagonal. For more details about reciprocal lattice and Brillouin zones, see Appendix \ref{chap:BZ}.  Figure \ref{fig:HS} shows the symmetry points inside the first Brillouin zone. The capital letters represent the high symmetry (HS) points inside the first Brillouin zone where their notations were traditionally used in the solid state physics literature \citep{Bouckaert1936}. Their  values are shown in Table \refeq{tab:HS}.  The different symmetry points of wavevectors correspond to the different kinds of irreducible representations of the space group \citep{Shmueli2001,Aroyo2006,PerezMato2011,Aroyo2014}. 

\begin{figure}[tbh!] 
	\centering
	\includegraphics[width=0.48\linewidth]{"images/rrl/hex"}
	\caption[The first Brillouin zone of a typical hexagonal Bravais lattice]{The first Brillouin zone of a typical hexagonal Bravais lattice. Figure taken from \citep{Setyawan2010}.}
	\label{fig:HS}
\end{figure}


\begin{table}[tbh!]
	\centering
	\caption{High symmetry points of an hexagonal Bravais lattice}
	\label{tab:HS}
	\begin{tabular}[t]{K{0.15\linewidth}K{0.1\linewidth}K{0.1\linewidth}K{0.1\linewidth}}
	\toprule
	\textbf{HS} & $\times \va{b}_1$ & $\times \va{b}_2$ & $\times \va{b}_3$ \\ \midrule
	$\Gamma$ & 0 & 0 & 0 \\
	$A$ & 0 & 0 & 1/2 \\
	$H$ & 1/3 & 1/3 & 1/2 \\
	$K$ & 1/3 & 1/3 & 0 \\
	$L$ & 1/2 & 0 & 1/2 \\
	$M$ & 1/2 & 0 & 0 \\ \bottomrule
	\end{tabular}
\end{table}

\clearpage


% \section{Crystallographic Directions and Planes}
text
\section{Photoluminescence Properties}
ZnO often exhibits green luminescence which has a peak around 2.4 and 2.5 eV.
\section{Defects}
add Kröger-Vink Notation for defects

consider degenerate semiconductors?

 