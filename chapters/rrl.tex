\chapter{Review of Related Literature}
% \section{Semiconductors}
%     \subsection{Properties}
%     \subsection{Applications of Semiconductors}
%     \subsection{Defects in Semiconductors}
\section{Properties of ZnO}
% \section{Zinc Oxide}
    describe ZnO in broad perspective
\section{Crystal Structure}
Consider different phases

\begin{figure}[tbh!] 
	\centering
	\includegraphics[width=0.5\linewidth]{"images/rrl/ZnO_unit"}
	\caption[Crystal structure of wurtzite \ch{ZnO} unit cell]{Crystal structure of wurtzite \ch{ZnO} unit cell}
	\label{fig:ZnO_unit}
\end{figure}

\section{Brillouin Zone Symmetry}
	The primitive vectors of the direct (real) hexagonal lattice are 
\begin{align}
	\va{a}_1 &= a \vu*{x}	\\
	\va{a}_2 &= \frac{a}{2} \vu*{x} + \frac{\sqrt{3}a}{2} \vu*{y} \\
	\va{a}_3 &= c \vu*{z}
\end{align}
where $a$ and $c$ are the lattice parameters of hexagonal \ch{ZnO}. On the other hand, the primitive vectors of the reciprocal lattice can be derived using the formula \eqref{eq:recipro_real} in Appendix \ref{chap:BZ} 

\begin{align}
	\va{b}_1 &= \frac{2 \pi}{a} \left(\vu*{k}_x - \frac{1}{\sqrt{3}} \vu*{k}_y \right) \\ 
	\va{b}_2 &= \frac{2 \pi}{a} \left( \frac{2}{\sqrt{3}} \vu*{k}_y \right) \\
	\va{b}_3 &= \frac{2 \pi}{a} \left( \frac{a}{c} \vu*{k}_z  \right)
\end{align}

The first Brillouin Zone of an hexagonal lattice is also an hexagonal. For more details about reciprocal lattice and Brillouin zones, see Appendix \ref{chap:BZ}.  Figure blah shows the symmetry points inside the first Brillouin

	
\section{Crystallographic Directions and Planes}

\section{Photoluminescence Properties}
\section{Defects}

