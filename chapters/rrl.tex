\chapter{Review of Related Literature} \label{chap:rrl}
% \section{Semiconductors}
%     \subsection{Properties}
%     \subsection{Applications of Semiconductors}
%     \subsection{Defects in Semiconductors}
\section{Properties of ZnO}
% \section{Zinc Oxide}
	describe ZnO in broad perspective
	donors and acceptors
	excitation in semiconductors
\section{Crystal Structure}
There are three possible crystal structures (phases) of \ch{ZnO}, namely: wurtzite, zinc blende, and rocksalt. The wurtzite structure belongs to the space group $C^4_{6v}$ in Schoenflies notation, $P6_3mc$ in the  Hermann–Mauguin notation, and space group number 186 in International Tables for Crystallography (ITA) notation \citep{Hahn2005}. The crystal structure of wurtzite ZnO is shown in Figure \ref{fig:ZnO_unit}. Each zinc atom is surrounded by four oxygen atoms, which are located
at the corners of an irregular tetrahedron.

\begin{figure}[tbh!] 
	\centering
	\includegraphics[width=0.7\linewidth]{"images/rrl/ZnO_unit"}
	\caption[Crystal structure of (a) wurtzite (b) zinc-blende and (c) rocksalt \ch{ZnO} unit cell]{Crystal structure of (a)wurtzite (b) zinc-blende and (c) rocksalt \ch{ZnO} unit cell. Figure taken from \citep{Ma2013}.}
	\label{fig:ZnO_unit}
\end{figure}

\section{Brillouin Zone Symmetry}
	The primitive vectors of the direct (real) hexagonal lattice are 
\begin{align}
	\va{a}_1 &= a \vu*{x}	\\
	\va{a}_2 &= \frac{a}{2} \vu*{x} + \frac{\sqrt{3}a}{2} \vu*{y} \\
	\va{a}_3 &= c \vu*{z}
\end{align}
where $a$ and $c$ are the lattice parameters of hexagonal \ch{ZnO}. On the other hand, the primitive vectors of the reciprocal lattice can be derived using the formula \eqref{eq:recipro_real} in Appendix \ref{chap:BZ} 

\begin{align}
	\va{b}_1 &= \frac{2 \pi}{a} \left(\vu*{k}_x - \frac{1}{\sqrt{3}} \vu*{k}_y \right) \\ 
	\va{b}_2 &= \frac{2 \pi}{a} \left( \frac{2}{\sqrt{3}} \vu*{k}_y \right) \\
	\va{b}_3 &= \frac{2 \pi}{a} \left( \frac{a}{c} \vu*{k}_z  \right)
\end{align}

The first Brillouin Zone of an hexagonal lattice is also an hexagonal. For more details about reciprocal lattice and Brillouin zones, see Appendix \ref{chap:BZ}.  Figure \ref{fig:HS} shows the symmetry points inside the first Brillouin zone. The capital letters represent the high symmetry (HS) points inside the first Brillouin zone where their notations were traditionally used in the solid state physics literature \citep{Bouckaert1936}. Their  values are shown in Table \refeq{tab:HS}.  The different symmetry points of wavevectors correspond to the different kinds of irreducible representations of the space group \citep{Shmueli2001,Aroyo2006,PerezMato2011,Aroyo2014}. 

\begin{figure}[tbh!] 
	\centering
	\includegraphics[width=0.48\linewidth]{"images/rrl/hex"}
	\caption[The first Brillouin zone of a typical hexagonal Bravais lattice]{The first Brillouin zone of a typical hexagonal Bravais lattice. Figure taken from \citep{Setyawan2010}.}
	\label{fig:HS}
\end{figure}



\begin{table}[tbh!]
	\centering
	\caption{High symmetry points of an hexagonal Bravais lattice}
	\label{tab:HS}
	\begin{tabular}[t]{K{0.15\linewidth}K{0.1\linewidth}K{0.1\linewidth}K{0.1\linewidth}}
	\toprule
	\textbf{HS} & $\times \va{b}_1$ & $\times \va{b}_2$ & $\times \va{b}_3$ \\ \midrule
	$\Gamma$ & 0 & 0 & 0 \\
	$A$ & 0 & 0 & 1/2 \\
	$H$ & 1/3 & 1/3 & 1/2 \\
	$K$ & 1/3 & 1/3 & 0 \\
	$L$ & 1/2 & 0 & 1/2 \\
	$M$ & 1/2 & 0 & 0 \\ \bottomrule
	\end{tabular}%
\end{table}

% \section{Crystallographic Directions and Planes}
% text
% \section{Photoluminescence Properties}
% ZnO often exhibits green luminescence which has a peak around 2.4 and 2.5 eV.

\section{Defects}
A perfect crystal never exists in nature. Atom arrangements do not follow perfect crystalline patterns since various factors such as temperature, pressure and chemical composition substantially affect the preferred structure. In addition, atoms are relatively immobile in a solid, hence, it is difficult to eliminate whatever imperfections introduced during crystal growth. It is this reason that defects or imperfections exist at stable conditions.  Crystal defects can be classified according to the dimension. The 0-dimensional defects affect localized points in the crystal site, thus they are called point defects. The one-dimensional defects are called dislocations. They are lines along  crystal lattice where the pattern is broken. The two-dimensional defects are surface defects, which include the external surface boundary and the grain boundaries along which crystals are joined together. Lastly, three-dimensional defects changes the lattice site at a finite volume. This includes precipitates, voids, and inclusion of second-phase particles. 

Among these types of defects, the point defects play a vital role in semiconductor engineering. Material properties are substantially altered upon modifications of point defects. Point defects can be classified into native (or intrinsic) and extrinsic defects. Native point defects are formed from the atoms of the host crystal. Extrinsic defects consist of impurity or foreign atoms. This includes doping, addition, and substitution. The point defect can be further subdivided to its type which includes the following  

\begin{itemize}
	\item Vacancies - the absence of an atom from its normal location in a perfect crystal structure
	\item Interstitials - an atom is occupying an interstitial site, a small void space that under ordinary circumstances is not occupied. Self-interstitials are form when the atom of a host crystal occupy the interstitial sites.
	\item Substitutionals - formed when an extra atom replaces a host atom
	\item Antisites - a specific kind of substitutionals in which a host atom occupies the site which was originally occupied by another type of host atom
	\item Frenkel Defects - an atom displaced from its position to a nearby interstitial site
	\item Schottky Defects - an equal  number  of  cations  and anions  are missing  from  their  lattice  sites. Hence,  electrical  neutrality  of  the  crystal is conserved. 

\end{itemize}

Insert figure

Vacancies, interstitals, Schottky, Frenkel, and antisite which do not involve the foreign atoms are called the native point defects. Otherwise, interstitals and substitutionals involving foreign atoms are called extrinsic point defects. A useful notation for dealing with point defects is the use of  Kröger-Vink notation \citep{Kroeger1964}. The general notation can  be expressed as  $X_Y^Z$ where $X$ is element symbol for an atomic species located on a site, else it is written $V$ for vacancy. The $Y$ is the type of site occupied by X; $i$ for an interstitial, otherwise, element symbol for site normally occupied by this element. Lastly $Z$ is the charge of defect normally written as $'$ or $-$ for negative charge, $\cdot$ or $+$ for positive charge, while $x$, $0$ or sometimes omitted for neutral charge. For instance, an oxygen antisite in ZnO crystal with a +2 charge state can be written as \ch{O}$_{\ch{Zn}}^{2+}$.

Point defects are 



% consider degenerate semiconductors?

 